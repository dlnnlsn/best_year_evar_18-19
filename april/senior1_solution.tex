\documentclass[a4paper, 12pt]{article}
%\documentclass{book}

% Important Packages:
 \usepackage{amsmath}    % need for subequations
 \usepackage{amsfonts}
 \usepackage{amsthm}
 \usepackage{graphicx}   % need for figures
 \usepackage{verbatim}   % useful for program listings
 \usepackage{tikz,tkz-euclide}
 \usepackage{amssymb}
 
 \usetikzlibrary{calc,patterns,angles,quotes}
\usetkzobj{all}

\def\deg{^{\circ}}
\newcommand\heading[1]{\ \\\large{\textbf{#1}}}
\newcommand\ora[1]{\overrightarrow{#1}}

\def\thm{Th\textsuperscript{\underline{m}}}

%------------------end---preamble--------------------
 
 % Useful macros 
 \def\tcb#1{\color{blue}{#1}}
 \def\tcr#1{\color{red}{#1}}	
 \def\tcg#1{\color{green}{#1}}
 \def\be{\begin{eqnarray}}	 	\def\ee{\end{eqnarray}}
 \def\bea{\begin{eqnarray}}	 	\def\eea{\end{eqnarray}}
 \def\bean{\begin{eqnarray*}}	\def\eean{\end{eqnarray*}}
 
 \def\D{\displaystyle}
 \def\T{\textstyle}
 \def\l{\left}
 \def\r{\right}
 \def\nf{n_{\!f}} % quark flavours
 \def\pa{\partial}
 \def\eg{e.\,g.}
 \def\ie{i.\,e.}

 \def\be{\begin{equation}}
 \def\ee{\end{equation}}
 \def\bea{\begin{eqnarray}}
 \def\eea{\end{eqnarray}}
 \def\bean{\begin{eqnarray*}}
 \def\eean{\end{eqnarray*}}
 \def\gsim{\mathrel{\rlap{\lower0.2em\hbox{$\sim$}}\raise0.2em\hbox{$>$}}}
 \def\ksim{\mathrel{\rlap{\lower0.2em\hbox{$\sim$}}\raise0.2em\hbox{$<$}}}
 \def\kg{\mathrel{\rlap{\lower0.25em\hbox{$>$}}\raise0.25em\hbox{$<$}}}
 
 \def\AA{${\buildrel_{\circ} \over {\mathrm{A}}}$}
 \def\bm#1{\mbox{\boldmath$#1$}}
 \newcommand{\eq}[1]{(\ref{#1})} 
 \def\pd{\partial}
 \def\d{\textrm{d}} 
 \def\T{\textstyle}
 \def\eg{e.\,g.}	% exempli gratia (for the sake of example)
 \def\ie{i.\,e.}	% id est (that is)


 % Page configuration:
 \topmargin -2.0cm
 \oddsidemargin -0.85cm
 \evensidemargin -0.85cm
 \textwidth 18cm
 \textheight 24cm
 
\begin{document}
\begin{center}
\textbf{April Camp 2019 \\ Senior Test 1} \\
\textbf{Solutions}
\end{center}
\vspace{5mm}

\begin{enumerate}

%  2018 shortlist A1
\item[1.]  \textit{Let $\mathbb{Q}_{> 0}$ denote the set of all positive rational numbers. Determine all functions $f: \mathbb{Q}_{> 0} \to \mathbb{Q}_{> 0}$ satisfying
$$
f \left( x^2 f(y)^2\right) = f(x)^2 f(y) $$
for all $x, y \in \mathbb{Q}_{> 0}$.}
\vspace{5mm}

\textbf{Answer:} $f(x) = 1$ for all $x \in \mathbb{Q}_{> 0}$

\textbf{Solution:} Take any $a, b \in \mathbb{Q}_{> 0}$. By substituting $x = f(a)$, $y = b$, and $x = f(b)$, $y = a$ into the given equation, we get
$$ f(f(a))^2 f(b) = f(f(a)^2 f(b)^2 ) = f(f(b))^2 f(a) $$
which yields
$$
\frac{f(f(a))^2}{f(a)} = \frac{f(f(b))^2}{f(b)} \qquad \textrm{ for all } a, b \in \mathbb{Q}_{> 0}
$$
In other words, this shows that there exists a constant $C \in \mathbb{Q}_{> 0}$ such that $f(f(a))^2 = C f(a)$, or
\begin{equation}  \label{a11}
    \left( \frac{f(f(a))}{C}  \right)^2 = \frac{f(a)}{C} \qquad \textrm{ for all } a \in \mathbb{Q}_{> 0}
\end{equation}

Denote by $f^n(x) = f(f(\dots(f(x))\dots))$ the $n$th iteration of $f$. Equality (\ref{a11}) yields
$$
\frac{f(a)}{C} = \left( \frac{f^2(a)}{C}  \right)^2 = \left( \frac{f^3(a)}{C}  \right)^4 = \dots = \left( \frac{f^{n+1}(a)}{C}  \right)^{2^n}
$$
for all positive integer $n$. So, $f(a)/C$ is the $2^n$-th power of a rational number for all positive integer $n$. This is impossible unless $f(a)/C = 1$, since otherwise the exponent of some prime in the prime decomposition of $f(a)/C$ is not divisible by sufficiently large powers of 2. Therefore, $f(a) = C$ for all $a \in \mathbb{Q}_{> 0}$.

Finally, after substituting $f \equiv C$ into the given condition, we get $C = C^3$, whence $C = 1$. So $f(x) \equiv 1$ is the unique function satisfying the given equation.
\qed


\vspace{5mm}
\item[2.]  \textit{Let $n > 1$ be a positive integer. Each cell of an $n \times n$ table contains an integer. Suppose that the following conditions are satisfied:}

\begin{enumerate}
    \item[(i)] \textit{Each number in the table is congruent to $1$ modulo $n$;}
    
    \item[(ii)] \textit{The sum of numbers in any row, as well as the sum of numbers in any column is congruent to $n$ modulo $n^2$.}
    
\end{enumerate}

 \textit{Let $R_i$ be the product of the numbers in the $i$-th row, and $C_j$ be the product of the numbers in the $j$-th column. Prove that the sums $R_1 + \dots + R_n$ and $C_1 + \dots + C_n$ are congruent modulo $n^4$.}
 \vspace{5mm}

\textbf{Proof: } Let $A_{i,j}$ be the entry in the $i$th row and the $j$th column; let $P$ be the product of all $n^2$ entries. For convenience, denote $a_{i, j} = A_{i,j}-1$ and $r_i = R_i - 1$. We show that
$$
\sum_{i=1}^n R_i \equiv (n-1) + P \quad (\textrm{mod } n^4).
$$
Due to symmetry of the problem conditions, the sum of all the $C_j$ is also congruent to $(n-1) + P$ modulo $n^4$, whence the conclusion.

By condition (i), the number $n$ divides $a_{i, j}$ for all $i$ and $j$. So, every product of at least two of the $a_{i,j}$ is divisible by $n^2$, hence
\begin{equation*}
    R_i = \prod_{j=1}^n (1 + a_{i,j}) = 1 + \prod_{j=1}^n a_{i,j} + \sum_{1 \leq j_1 < j_2 \leq n} a_{i, j_1} a_{i, j_2} + \dots \equiv 1 - n + \sum_{j=1}^n A_{i,j} \quad (\textrm{mod } n^2)
\end{equation*}
for every index $i$. Using condition (ii), we obtain $R_i \equiv 1$ (mod $n^2$), and so $n^2 \,|\, r_i$.

Therefore, every product of at least two of the $r_i$ is divisible by $n^4$. Repeating the same argument, we obtain
$$
P = \prod_{i=1}^n R_i = \prod_{i=1}^n (1 + r_i) \equiv 1 + \sum_{i=1}^n r_i \quad (\textrm{mod } n^4)
$$
whence
$$
\sum_{i=1}^n R_i = n + \sum_{i=1}^n r_i \equiv n + (P-1) \quad (\textrm{mod } n^4)
$$
as desired.

\qed


\vspace{5mm}
\item[3.]  \textit{A point $T$ is chosen inside a triangle $ABC$. Let $A_1$, $B_1$, and $C_1$ be the reflections of $T$ in $BC$, $CA$, and $AB$, respectively. Let $\Omega$ be the circumcircle of the triangle $A_1 B_1 C_1$. The lines $A_1 T$, $B_1 T$, and $C_1 T$ meet $\Omega$ again at $A_2$, $B_2$, and $C_2$, respectively. Prove that the lines $AA_2$, $BB_2$, and $CC_2$ are concurrent on $\Omega$.}
 \vspace{5mm}

\textbf{Proof: } By $\sphericalangle (\ell, n)$ we always mean the directed angle of the lines $\ell$ and $n$, taken modulo $180^\circ$.

Let $CC_2$ meet $\Omega$ again at $K$ (as usual, if $CC_2$ is tangent to $\Omega$, we set $T = C_2$). We show that the line $BB_2$ contains $K$; similarly, $AA_2$ will also pass through $K$. For this purpose, it suffices to prove that
\begin{equation} \label{g41}
    \sphericalangle (C_2C, C_2A_1) = \sphericalangle (B_2 B, B_2A_1)
\end{equation}

By the problem condition, $CB$ and $CA$ are the perpendicular bisectors of $TA_1$ and $TB_1$, respectively. Hence $C$ is the circumcentre of the triangle $A_1 T B_1$. Therefore,
$$
\sphericalangle (CA_1, CB) = \sphericalangle (CB, CT) = \sphericalangle (B_1 A_1, B_1T) = \sphericalangle (B_1 A_1, B_1 B_2).
$$
In circle $\Omega$, we have $\sphericalangle (B_1 A_1, B_1 B_2) = \sphericalangle (C_2 A_1, C_2 B_2)$. Thus,
$$
\sphericalangle(CA_1, CB) = \sphericalangle(B_1 A_1, B_1 B_2) = \sphericalangle(C_2 A_1, C_2 B_2).
$$
Similarly, we get
$$
\sphericalangle (BA_1, BC) = \sphericalangle (C_1 A_1, C_1 C_2) = \sphericalangle (B_2 A_1, B_2 C_2).
$$
The two obtained relations yield that the triangles $A_1 BC$ and $A_1 B_2 C_2$ are similar and equioriented, hence
$$
\frac{A_1 B_2}{A_1 B} = \frac{A_1 C_2}{A_1 C} \quad 
\textrm{ and } \quad 
\sphericalangle(A_1 B, A_1 C) = \sphericalangle (A_1 B_2, A_1 C_2).
$$

The second equality may be rewritten as $\sphericalangle(A_1 B, A_1 B_2) = \sphericalangle (A_1 C, A_1 C_2)$, so the triangles $A_1 B B_2$ and $A_1 C C_2$ are also similar and equioriented. This establishes (\ref{g41}).


\begin{figure}[h]
    \centering
    \includegraphics[width = 0.7\textwidth]{2018_G4}
\end{figure}


\qed
\vspace{6mm}


    

\end{enumerate}
\end{document}




