\documentclass[a4paper, 12pt]{article}
\usepackage{amsmath,amsfonts}
\usepackage{amssymb,amsthm}
\usepackage{tikz,tkz-euclide}
\usepackage{fullpage}
\usepackage{fancyvrb}

\usetikzlibrary{calc,patterns,angles,quotes}
\usetkzobj{all}

\title{Senior Test 4}
\author{April Camp 2019}
\date{Time: $4 \frac{1}{2}$ hours}

\begin{document} \maketitle \vspace{24pt}

\begin{enumerate}

%  2018 shortlist G2
\item[1.]  Let $ABC$ be a triangle with $AB = AC$, and let $M$ be the midpoint of $BC$. Let $P$ be a point such that $PB < PC$ and $PA$ is parallel to $BC$. Let $X$ and $Y$ be points on the lines $PB$ and $PC$, respectively, so that $B$ lies on the segment $PX$, $C$ lies on the segment $PY$, and $\angle PXM = \angle PYM$. Prove that the quadrilateral $APXY$ is cyclic.

\vspace{24pt}


%  2018 shortlist C3
\item[2.]  Let $n$ be a given positive integer.  Sisyphus performs a sequence of turns on a board consisting of $n+1$ squares in a row, numbered from 0 to $n$ from left to right.  Initially, $n$ stones are put into square 0, and the other squares are empty. At every turn, Sisyphus chooses an nonempty square, say with $k$ stones, takes one of those stones and moves it to the right by at most $k$ squares (the stone should stay within the board). Sisyphus' aim is to move all $n$ stones to square $n$.

Prove that Sisyphus cannot reach the aim in less than
 $$\left\lceil \frac{n}{1} \right\rceil + \left\lceil \frac{n}{2} \right\rceil +\left\lceil \frac{n}{3} \right\rceil + \dots + \left\lceil \frac{n}{n} \right\rceil
$$
turns. (As usual, $\lceil x \rceil$ stands for the least integer not smaller than $x$.)


\vspace{24pt}
%  2018 shortlist N5
\item[3.]   Four positive integers $x, y, z$, and $t$ satisfy the relations
$$ xy - zt = x + y = z + t. $$
Is it possible that both $xy$ and $zt$ are perfect squares?




\end{enumerate}

\vfill

\centering
%\begin{BVerbatim}
%     |\_/|                  
%     | @ @   Woof! 
%     |   <>              _  
%     |  _/\------____ ((| |))
%     |               `--' |   
% ____|_       ___|   |___.' 
%/_/_____/____/_______|
%\end{BVerbatim}

\vspace{12mm}


\end{document}
