\documentclass[a4paper, 12pt]{article}
%\usepackage[margin=2cm]{geometry} 
\usepackage{amsmath,amsfonts}
\usepackage{amssymb,amsthm}
\usepackage{tikz,tkz-euclide}
\usepackage{fullpage}
\usepackage{fancyvrb}

\usetikzlibrary{calc,patterns,angles,quotes}
\usetkzobj{all}

\title{Senior Test 1}
\author{April Camp 2019}
\date{Time: $4 \frac{1}{2}$ hours}

\begin{document} \maketitle

\begin{enumerate}

%  2018 shortlist A1
\item[1.]  Let $\mathbb{Q}_{> 0}$ denote the set of all positive rational numbers. Determine all functions $f: \mathbb{Q}_{> 0} \to \mathbb{Q}_{> 0}$ satisfying

$$ f \left( x^2 f(y)^2\right) = f(x)^2 f(y) $$
for all $x, y \in \mathbb{Q}_{> 0}$. \\

\vspace{20pt}

%  2018 shortlist C1
\item[2.]    
%Let $n \geq 3$ be an integer. Prove that there exists a set $S$ of $2n$ positive integers satisfying the following property: For every $m = 2, 3, \dots, n$ the set $S$ can be partitioned into two subsets with equal sums of elements, with one of the subsets of cardinality $m$. \\

% 2018 shortlist N2
Let $n > 1$ be a positive integer. Each cell of an $n \times n$ table contains an integer. Suppose that the following conditions are satisfied:

\begin{enumerate}
    \item[(i)] Each number in the table is congruent to $1$ modulo $n$;
    
    \item[(ii)] The sum of numbers in any row, as well as the sum of numbers in any column is congruent to $n$ modulo $n^2$.
    
\end{enumerate}

Let $R_i$ be the product of the numbers in the $i$-th row, and $C_j$ be the product of the numbers in the $j$-th column. Prove that the sums $R_1 + \dots + R_n$ and $C_1 + \dots + C_n$ are congruent modulo $n^4$.


\vspace{20pt}

% 2018 shortlist G2
\item[3.]   
%Let $ABC$ be a triangle with $AB=AC$, and let $M$ be the midpoint of $BC$. Let $P$ be a point such that $PB < PC$ and $PA$ is parallel to $BC$. Let $X$ and $Y$ be points on the lines $PB$ and $PC$, respectively, such that $B$ lies on the segments $PX$, $C$ lies on the segment $PY$, and $\angle PXM = \angle PYM$. Prove that the quadrilateral $APXY$ is cyclic.

% 2018 shortlist G4
A point $T$ is chosen inside a triangle $ABC$. Let $A_1$, $B_1$, and $C_1$ be the reflections of $T$ in $BC$, $CA$, and $AB$, respectively. Let $\Omega$ be the circumcircle of the triangle $A_1 B_1 C_1$. The lines $A_1 T$, $B_1 T$, and $C_1 T$ meet $\Omega$ again at $A_2$, $B_2$, and $C_2$, respectively. Prove that the lines $AA_2$, $BB_2$, and $CC_2$ are concurrent on $\Omega$.

\end{enumerate}

\vfill

\centering
%\begin{BVerbatim}
%     |\_/|                  
%     | @ @   Woof! 
%     |   <>              _  
%     |  _/\------____ ((| |))
%     |               `--' |   
% ____|_       ___|   |___.' 
%/_/_____/____/_______|
%\end{BVerbatim}

\vspace{12mm}


\end{document}
