\documentclass[a4paper, 12pt]{article}
% \usepackage[margin=2cm]{geometry}
\usepackage{amsmath,amsfonts}
\usepackage{amssymb,amsthm}
\usepackage{tikz,tkz-euclide}
\usepackage{fullpage}
\usepackage{fancyvrb}

\usetikzlibrary{calc,patterns,angles,quotes}
\usetkzobj{all}

\title{PAMO Stream Test 1}
\author{April Camp 2019}
\date{Time: $4 \frac{1}{2}$ hours}

\begin{document} \maketitle

\begin{enumerate}

\item\label{problem:geometry_common_point} % SW-2009-5
In a triangle $ABC$, let $D$ and $E$ be the midpoints of $AB$ and $AC$, respectively, and let $F$ be the foot of the altitude through $A$. Show that the line $DE$, the angle bisector of $\angle ACB$ and the circumcircle of $ACF$ pass through a common point.


\item Let $f(n) = n + \lfloor \sqrt{n} \rfloor$. Prove that for every positive integer $m$, the integer sequence $m$, $f(m)$, $f(f(m))$, $\dots$ contains at least one square of an integer.


\item A game is played on an $m \times n$ chessboard. At the beginning, there is a coin on one of the squares. Two players take turns to move the coin to an adjacent square (horizontally or vertically). The coin may never be moved to a square that has been occupied before. If a player cannot move any more, he loses. Prove:
\begin{enumerate}

\item If the size (number of squares) of the board is even, then the player to move first has a winning strategy, regardless of the initial position.

\item If the size of the board is odd, then the player to move first has a winning strategy if and only if the coin is initially placed on a square whose colour is not the same as the colour of the corners.

\end{enumerate}


\end{enumerate}

\vfill

\centering
%\begin{BVerbatim}
%     |\_/|                  
%     | @ @   Woof! 
%     |   <>              _  
%     |  _/\------____ ((| |))
%     |               `--' |   
% ____|_       ___|   |___.' 
%/_/_____/____/_______|
%\end{BVerbatim}

\vspace{12mm}


\end{document}
