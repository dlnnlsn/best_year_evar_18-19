%-----------------begin---preamble-------------------
\documentclass{article}
\usepackage{amsmath,amssymb,amsthm}
\usepackage{tikz,tkz-euclide}
\usepackage{marginnote}
\usepackage{float}
\usepackage[margin=0.8in]{geometry}
% \usepackage{enumerate}


\usetikzlibrary{calc,patterns,angles,quotes}
\usetkzobj{all}

\def\deg{^{\circ}}
\def\thm{Th\textsuperscript{\underline{m}}}
\newcommand\heading[1]{\ \\\large{\textbf{#1}}}
\newcommand\ora[1]{\overrightarrow{#1}}

\newtheorem*{problem}{Problem}
%------------------end---preamble--------------------

\begin{document}
	\begin{problem}[BMO Round 2, 2018]
		Consider triangle $ABC$. Let $M$ be the midpoint of $AC$. The circle tangent to $BC$ at $B$ and passing through $M$ meets the line $AB$ again at $P$. Prove that $AB \cdot BP =2BM^2$.
	\end{problem}
	\begin{proof}
		\begin{figure}[htb!]
		\begin{center}
		\begin{tikzpicture}[scale=3]
			% \useasboundingbox (-2,-2) rectangle  (2,2);
			\tkzDefPoint(-1.5,2){A}
			\tkzDefPoint(0,0){B}
			\tkzDefPoint(2.1,0){C}
			\tkzDefBarycentricPoint(C=1,A=1)\tkzGetPoint{M}
			\tkzDefLine[orthogonal=through B](B,C)\tkzGetPoint{h1}
			\tkzDefLine[mediator](B,M)\tkzGetPoints{m1}{m2}
			\tkzInterLL(m1,m2)(B,h1)\tkzGetPoint{O}
			\tkzInterLC(A,B)(O,B)\tkzGetSecondPoint{P}
			\tkzDefBarycentricPoint(B=1,A=1)\tkzGetPoint{N}
			\tkzDrawSegments(A,B B,C C,A M,P M,N B,M)
			\tkzDrawCircle(O,B)
			\tkzDrawPoints[fill=black!100](A,B,C,M,N,P)
			\tkzLabelPoints[below](B,C)
			\tkzLabelPoints[left](A,P,N)
			\tkzLabelPoints[right](M)
			\tkzMarkSegments[mark=>>](N,M B,C)
			\end{tikzpicture}
		\end{center}		
		\end{figure}

	Construct point $N$, the midpoint of $AB$. By midpoint \thm, $NM||BC$. Using this and Tan-Chord \thm, we get: $\angle BMN=\angle MBC=\angle MPB$. Since we also have that $\angle PBM=\angle MBN$, we have that $\triangle BMN \sim \triangle BPM$. So
	\begin{flalign}
		\frac{BM}{BP}&=\frac{BN}{BM}\nonumber\\
		\Rightarrow BM^2&=BN \cdot BP=\left(\frac{1}{2}AB\right)\cdot BP\nonumber
	\end{flalign}
	\end{proof}

	
	
\end{document}