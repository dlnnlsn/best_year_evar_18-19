\documentclass[a4paper, 12pt]{article}
%\documentclass{book}

% Important Packages:
 \usepackage{amsmath}    % need for subequations
 \usepackage{amsfonts}
 \usepackage{amsthm}
 \usepackage{graphicx}   % need for figures
 \usepackage{verbatim}   % useful for program listings
 %\usepackage{subfig}  % use for side-by-side figures
 %\usepackage{wrapfig}
 %\usepackage{listings}	 % creates code blocks
 %\usepackage[colorlinks=true]{hyperref}   % use for hypertext links, including
                     % those to external documents and URLs
 %\usepackage{multirow}
 %\usepackage{tikz}
 %\usepackage{enumerate}
 %\usetikzlibrary{decorations.pathreplacing,decorations.pathmorphing}
 %\usetikzlibrary{calc}
 %\usepackage[colorinlistoftodos]{todonotes}
 \usepackage{tikz,tkz-euclide}
 
 \usetikzlibrary{calc,patterns,angles,quotes}
\usetkzobj{all}

\def\deg{^{\circ}}
\newcommand\heading[1]{\ \\\large{\textbf{#1}}}
\newcommand\ora[1]{\overrightarrow{#1}}

\def\thm{Th\textsuperscript{\underline{m}}}

%------------------end---preamble--------------------
 
 % Useful macros 
 \def\tcb#1{\color{blue}{#1}}
 \def\tcr#1{\color{red}{#1}}	
 \def\tcg#1{\color{green}{#1}}
 \def\be{\begin{eqnarray}}	 	\def\ee{\end{eqnarray}}
 \def\bea{\begin{eqnarray}}	 	\def\eea{\end{eqnarray}}
 \def\bean{\begin{eqnarray*}}	\def\eean{\end{eqnarray*}}
 
 \def\D{\displaystyle}
 \def\T{\textstyle}
 \def\l{\left}
 \def\r{\right}
 \def\nf{n_{\!f}} % quark flavours
 \def\pa{\partial}
 \def\eg{e.\,g.}
 \def\ie{i.\,e.}

 \def\be{\begin{equation}}
 \def\ee{\end{equation}}
 \def\bea{\begin{eqnarray}}
 \def\eea{\end{eqnarray}}
 \def\bean{\begin{eqnarray*}}
 \def\eean{\end{eqnarray*}}
 \def\gsim{\mathrel{\rlap{\lower0.2em\hbox{$\sim$}}\raise0.2em\hbox{$>$}}}
 \def\ksim{\mathrel{\rlap{\lower0.2em\hbox{$\sim$}}\raise0.2em\hbox{$<$}}}
 \def\kg{\mathrel{\rlap{\lower0.25em\hbox{$>$}}\raise0.25em\hbox{$<$}}}
 
 \def\AA{${\buildrel_{\circ} \over {\mathrm{A}}}$}
 \def\bm#1{\mbox{\boldmath$#1$}}
 \newcommand{\eq}[1]{(\ref{#1})} 
 \def\pd{\partial}
 \def\d{\textrm{d}} 
 \def\T{\textstyle}
 \def\eg{e.\,g.}	% exempli gratia (for the sake of example)
 \def\ie{i.\,e.}	% id est (that is)


 % Page configuration:
 \topmargin -2.0cm
 \oddsidemargin -0.85cm
 \evensidemargin -0.85cm
 \textwidth 18cm
 \textheight 24cm
 
\begin{document}
\begin{center}
\textbf{Stellenbosch Camp December 2018 \\ Senior Test 4} \\
\textbf{Solutions}
\end{center}
\vspace{5mm}

\begin{enumerate}
    
\item[1.] 


\qed
\vspace{5mm}


\item[2.]
\qed


\item[3.]
 \qed
\vspace{5mm}

\item[4.]  Let $x = a^{2/3}, y = b^{2/3}$ and $z = c^{2/3}$. By a well-known special case of Schur's inequality, we have
\begin{equation*}
    x^3  + y^3 + z^2 + 3xyz \geq xy(x+y) + yz(y+z) + zx(z+x)
\end{equation*}
By AM-GM, we also have
\begin{equation*}
    xy(x+y) = x^2y + xy^2 \geq 2 \sqrt{x^2 y \cdot xy^2} = 2 x^{3/2} y^{3/2}
\end{equation*}
and similarly for $yz(y+z)$ and $zx(z+x)$. We therefore obtain
\begin{align*}
    a^2 + b^2 + c^2 + 3 &= x^3 + y^3 + z^3 + 3xyz \\
    &\geq xy(x+y) + yz(y+z) + zx(z+x) \\
    &\geq 2 x^{3/2} y^{3/2} + 2 y^{3/2} z^{3/2} + 2 z^{3/2} x^{3/2} \\
    &= 2(ab + bc + ca)
\end{align*}
which proves the inequality. \qed
\vspace{5mm}

\item[5.]  Assume for contradiction the problem claim does not hold, and let $p$ be the smallest (odd) prime not appearing in the sequence. We thus have that $p$ is not the least prime factor of 
$$1 + n \prod_{i=1}^n p_i^{i!}$$
for all $i$. Let $C$ be the smallest integer such that all primes less than $p$ appear in $p_1, p_2, \dots, p_C$. Thus, if $p$ divides $1 + n \prod_{i=1}^n p_i^{i!}$ for some $n > C$, then it must be the smallest factor.

Therefore, $p$ does not divide $1 + n \prod_{i=1}^n p_i^{i!}$ for all $i > C$. Now, define $T_m = \prod_{i=1}^m p_i^{i!}$. By Fermat's Little Theorem, we have for sufficiently high $i$ and $p_i$ (indeed, for $i > \textrm{max}(C, p-1)$), $p_i^{i!} \equiv 1$ (mod $p$). Thus, the residuce class of $T_m$ stays constant for sufficiently high $m$. Therefore, there exists an $m$ high enough such that $m T_m \equiv -1$ (mod $p$), noting that $T_m$ is not divisible by $p$. This proves that $p$ divides $1 + m T_m$, which is a contradiction. \qed


    

\end{enumerate}
\end{document}




