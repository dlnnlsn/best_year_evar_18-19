\documentclass{article}

\usepackage{verbatim}
\usepackage{amsmath,amsthm,amsfonts}
\usepackage{enumerate}
\usepackage{fullpage}
\usepackage{fancyvrb}

\newcommand{\ds}{\ensuremath{\displaystyle}}

\title{Intermediate Test 2 -- Solutions}
\author{Stellenbosch Camp 2018}
\date{\vspace{-12pt}}


\begin{document}

\maketitle

\begin{enumerate}

\item %The Aether
\textit{The magical country of Vertexia, there are a certain number of cities. Each pair of cities is connected by exactly one magical portal, and it turns out that there are 5050 portals in total. How many cities are there in this wonderful land?}

Let the number of cities in Vertexia be $c$. Then since each pair of cities is connected by exactly one portal, the number of portals is equal to the number of pairs of cities, which is $\binom{c}{2} = \frac{c(c-1)}{2}$. Thus \[ \frac{c(c-1)}{2} = 5050 \iff c^2 -c = 2\times5050 = 10100 \iff 0 = c^2 -c -10100 = (c-101)(c+100) \]
\[ \iff c = 101 \quad \mathrm{since} \quad c > 0. \]


\vspace{12pt}
\item 
\textit{Let $M$ be the midpoint of side $AB$ of equilateral triangle $ABC$, and let points $N$, $S$ and $K$ divide side $BC$ into four equal segments. Given that $P$ is the midpoint of $CM$, prove that $\angle MNB = \angle KPN = 90^\circ$.}

Draw segments $SP$ and $SM$. Since $\frac{BS}{BC} = \frac{1}{2} = \frac{BM}{BA}$, by SAS similarity we have $\triangle MSB \sim \triangle ACB$. Hence $\triangle MSB$ is an equilateral triangle. Since $MN$ is a median of an equilateral triangle, it is also an altitude; thus $\angle MNB = 90^\circ$. Now, since $\frac{CP}{CM} = \frac{CS}{CB}$, we have that $\triangle PCS \sim \triangle MCB$ by SAS similarity. Thus $SP = \frac{BM}{2} = \frac{AB}{4} = SK = SN$. Hence, $SP$ is a median of $\triangle KPN$ which is half $KN$, so $\angle KPN = 90^\circ$.


\vspace{12pt}
\item %The Phil
\textit{Find all real numbers $x$ such that \[ (x+3)^4 + (x+5)^4 = 4. \]}

Let $y = x + 4$. Then we wish to solve ${(y - 1)}^4 + {(y + 1)}^4 = 4$. Using the binomial expansion, this becomes
\[
	2y^4 + 12y^2 + 2 = 4
\]
which is equivalent to
\[
	{(y^2)}^2 + 6y^2 - 1 = 0. 
\]

Using the binomial formula, we find that
\[
	y^2 = \frac{-6 \pm \sqrt{6^2 + 4}}{2} = -3 \pm \sqrt{10}.
\]

Since $y$ is real, $y^2$ is a positive real number, and so $y^2 = -3 + \sqrt{10}$, giving us that $x = \sqrt{-3 + \sqrt{10}} - 4$, or $x = -\sqrt{-3 + \sqrt{10}} - 4$.


\vspace{12pt}
\item %
\textit{Find all positive integers $m, n$ such that \[ 1 + 5 \cdot 2^m = n^2. \]}

\[ 5 \cdot 2^m = (n+1)(n-1) \]
Let m = p + q. We have two cases to consider.
 \[ \mathrm{Case} \enskip 1: \quad n + 1 = 5 \cdot 2^p \enskip \mathrm{and} \enskip n - 1 = 2^q
 \iff n = 5 \cdot 2^p - 1 = 2^q + 1
 \iff 5 \cdot 2^{p-1} - 1 = 2^{q-1} \]
Since $5 \cdot 2^{p-1} - 1$ is odd and $2^{q-1}$ is even for all $p, q > 1$, the only potential solutions are when $p = 1$ or $q = 1$. Checking these, $p = 1$, $q = 3$ and thus $(m, n) = (4, 9)$ is a solution. Similarly,
\[ \mathrm{Case} \enskip 2: \quad n - 1 = 5 \cdot 2^p \enskip \mathrm{and} \enskip n + 1 = 2^q 
\iff 5 \cdot 2^{p-1} = 2^{q-1} - 1, \] yields no solutions for $p = 1$ or $q = 1$. Thus the only solution is $(m, n) = (4, 9)$.


\vspace{12pt}
\item % Problem 2.17 (Russia 1972)
\textit{Show that if we are given 50 segments on the real line, then either there are 8 of them which are pairwise disjoint or 8 of them with a common point.}

\textit{A segment is defined as a closed interval $[a, b] = \{x \in \mathbb{R} \;:\; a \leq x \leq b \}$.}

Let $[a_1, b_1]$ be the segment with the smallest right endpoint. If more than 7 segments contain $b_1$, then we are finished. If this number is $\leq 7$, then at least 43 segments lie completely to the right of $b_1$. From these segments, select $[a_2, b_2]$ with the smallest right endpoint. Then either $b_2$ belongs to 8 segments, or there exist 36 segments to the right of $b_2$.

Continuing in this way, either we find a point belonging to eight segments, or we find seven pairwise disjoint segments $[a_1, b_1], [a_2, b_2], \dots, [a_7, b_7]$ such that to the right of $[a_k, b_k]$ lie at least $(50-7k)$ segments, i.e. to the right of $[a_7, b_7]$ lies at least one segment $[a_8, b_8]$. \qed 


\end{enumerate}


\vspace{12pt}
\begin{center}
\begin{BVerbatim}
  __
  (\,--------'()'--o
   (_    ___    /~"
    (_)_)  (_)_)
\end{BVerbatim}
\end{center}

\end{document}
