\documentclass{article}

\usepackage{verbatim}
\usepackage{amsmath,amsfonts}
\usepackage{enumerate}
\usepackage{fancyvrb}
\usepackage{graphicx}
\usepackage[cm]{fullpage}
\usepackage{float}

\newcommand{\ds}{\ensuremath{\displaystyle}}
\newcommand{\floor}[1]{\ensuremath{\left\lfloor#1\right\rfloor}}
\newcommand{\fracpart}[1]{\ensuremath{\left\{#1\right\}}}

\title{Intermediate Test 4 -- Solutions}
\author{Stellenbosch Camp 2018}
\date{\vspace{-12pt}}


\begin{document}

\maketitle

\begin{enumerate}[1.]

\item % 

\vspace{12pt}
\item 



\vspace{12pt}
\item %



\vspace{6pt}
\item % Thanks to Andrew
\textit{Consider two circles $\Gamma_1$ and $\Gamma_2$ that intersect at points A and B. Let $l$ be a line tangent to circles $\Gamma_1$ and $\Gamma_2$ at $S$ and $T$, respectively. Lines $AB$ and $ST$ intersect at point $M$. Furthermore line $BT$ intersect circle $\Gamma_1$ again at point $R$. Let the intersection of $MR$ and $SB$ be $X$ and the intersection of $TX$ and $RS$ be $C$. 
Prove that $CB$ and $ST$ are parallel.}

First notice that from Power of a Point theorem with M as point with respect to circle $\Gamma_1$ we have that $MS^2 = MB \times MA$. Similarly, $MT^2 = MB \times MA$. Thus we have that $MS = MT$. Furthermore $\triangle RST$ has three concurrent cevians $RM$, $SB$ and $TC$ intersecting at $X$. Thus from Ceva's Theorem we have that $$\frac{RC}{CS} \times \frac{SM}{MT} \times \frac{TB}{BR} = 1$$
Since $\frac{SM}{MT} = 1$, this simplifies to $\frac{RC}{CS} = \frac{BR}{BT}$. From the triangle proportionality theorem, it is clear that $CB$ is parallel to $ST$.

\vspace{6pt}
\item %




\vspace{6pt}
\item %





\end{enumerate}


\vfill
\centering
\begin{BVerbatim}
 ,    /-.
((___/ __>
/      }
\ .--.(    ___
 \\   \\  /___\
\end{BVerbatim}

\end{document}
