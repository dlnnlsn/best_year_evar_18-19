\documentclass{article}

\usepackage{verbatim}
\usepackage{amsmath,amsfonts}
\usepackage{enumerate}
\usepackage{fancyvrb}
\usepackage{graphicx}
\usepackage[cm]{fullpage}
\usepackage{float}

\newcommand{\ds}{\ensuremath{\displaystyle}}
\newcommand{\floor}[1]{\ensuremath{\left\lfloor#1\right\rfloor}}
\newcommand{\fracpart}[1]{\ensuremath{\left\{#1\right\}}}

\title{Intermediate Test 4 -- Solutions}
\author{Stellenbosch Camp 2018}
\date{\vspace{-12pt}}


\begin{document}

\maketitle

\begin{enumerate}[1.]

\item[1.]  We claim there are 2001 lockers. Indeed, the total cost comes to:

\begin{equation*}
    3c \cdot (9 - 0) + 6c \cdot (99 - 9) + 9c \cdot (999 - 99) + 12c \cdot (2001 - 999) = 20691c
\end{equation*}
\qed
\vspace{5mm}


\item[2.]   

\begin{enumerate}
    \item[(a)]  If $x$ is prime, then the unique prime factorisation of $y$ can only consist of the prime $x$. Thus $y = x^k$ for some $k \geq 1$. This gives the equation
    \begin{equation*}
        x^{2018} = x^{kx}
    \end{equation*}
    which has solutions when $2018 = kx$. We note the prime factorisation of $2018$ is $2018 = 2 \cdot 1009$, which therefore yields two solutions: $(x, y) = (2, 2^{1009})$ and $(x, y) = (1009, 1009^2)$.
    
    \item[(b)]  Note that $(x, y) = (1, 1)$ yields a valid solution. Now, assume $x, y \geq 2$. Thus $x = a^m$ and $y = a^n$ where $a \geq 2$ and $m, n \geq 1$. Substituting this into the equation gives:
    \begin{equation*}
        2018m = a^m \cdot n
    \end{equation*}
    Considering the case $m = 1$, yields only one additional solution: $(x, y) = (2018, 2018)$. The case $m = 2$ yields $(x, y) = (2^2, 2^{1009})$. Note that $m \geq 3$ yields no further solutions, as the exponent grows faster than $2018m$. Thus the solutions are
    \begin{equation*}
        (x, y) \in \{(1, 1), (2, 2^{1009}), (1009, 1009^2), (2018, 2018), (2^2, 2^{1009}) \}
    \end{equation*}
    
    
\end{enumerate}

\qed


\vspace{12pt}
\item %



\vspace{6pt}
\item % Thanks to Andrew
\textit{Consider two circles $\Gamma_1$ and $\Gamma_2$ that intersect at points A and B. Let $l$ be a line tangent to circles $\Gamma_1$ and $\Gamma_2$ at $S$ and $T$, respectively. Lines $AB$ and $ST$ intersect at point $M$. Furthermore line $BT$ intersect circle $\Gamma_1$ again at point $R$. Let the intersection of $MR$ and $SB$ be $X$ and the intersection of $TX$ and $RS$ be $C$. 
Prove that $CB$ and $ST$ are parallel.}

First notice that from Power of a Point theorem with M as point with respect to circle $\Gamma_1$ we have that $MS^2 = MB \times MA$. Similarly, $MT^2 = MB \times MA$. Thus we have that $MS = MT$. Furthermore $\triangle RST$ has three concurrent cevians $RM$, $SB$ and $TC$ intersecting at $X$. Thus from Ceva's Theorem we have that $$\frac{RC}{CS} \times \frac{SM}{MT} \times \frac{TB}{BR} = 1$$
Since $\frac{SM}{MT} = 1$, this simplifies to $\frac{RC}{CS} = \frac{BR}{BT}$. From the triangle proportionality theorem, it is clear that $CB$ is parallel to $ST$.

\vspace{6pt}
\item %




\vspace{6pt}
\item %





\end{enumerate}


\vfill
\centering
\begin{BVerbatim}
 ,    /-.
((___/ __>
/      }
\ .--.(    ___
 \\   \\  /___\
\end{BVerbatim}

\end{document}
