\documentclass{article}

\usepackage{verbatim}
\usepackage{amsmath,amsfonts}
\usepackage{enumerate}
\usepackage{fancyvrb}

\newcommand{\ds}{\ensuremath{\displaystyle}}
\newcommand{\floor}[1]{\ensuremath{\left\lfloor#1\right\rfloor}}
\newcommand{\fracpart}[1]{\ensuremath{\left\{#1\right\}}}

\title{Beginner Test 3}
\author{Stellenbosch Camp 2018}
\date{Time: $4$ hours}


\begin{document}

\maketitle

\begin{enumerate}[1.]

\vspace{6pt}
\item 
Let $AB$ be a chord in a circle with centre $O$, and let $C$ be a point on the larger arc $AB$. Show that $\angle AOB = 2\angle ACB$.


\vspace{6pt}
\item 
How many different rearrangments are there of the word TARTAGLIA?


\vspace{6pt}
\item 
Consider a game wherein two players Emma and Dylan take turns to take between 1 and 7 stones, inclusive, from a pile which starts with 2018 stones. If Emma plays first, does one of the players have a winning strategy, and if so what is it?


\vspace{6pt}
\item 
In the figure $ABC$ is a tangent to the circumscribed circle of $\triangle PBG$. $PS$ and $DG$ are both parallel to $ABC$. Chords $BP$ and $BS$ cut $DG$ at $E$ and $F$ respectively. Prove that:
\begin{enumerate}[a.]
\item
$\angle G_1 = \angle P_1$
\item
$\triangle BGP$ is similar to $\triangle BEG$
\item
$BG^2 = BP \times BE$
\item
$\frac{BG^2}{BP^2} = \frac{BF}{BS}$

\end{enumerate}


\vspace{6pt}
\item 


\vspace{6pt}
\item 


\vspace{6pt}
\item 


\vspace{6pt}
\item 


\vspace{6pt}
\item % 
The student lockers at Olympic High are numbered consecutively beginning with locker number 1. The plastic digits used to number the lockers cost $3$ cents per piece. Thus, it costs $3$ cents to number locker $9$ and $6$ cents to number locker $42$. If it costs R206.91 to label all the lockers, how many lockers are there at the school?


\vspace{6pt}
\item % UK IMO 2001 IMO training problem sheet
Given the equation $x^{2018} = y^x$,
\begin{enumerate}
  \item find all pairs $(x,y)$ of solutions with $x$ prime and $y$ a positive integer;
  \item find all pairs $(x,y)$ of positive integers satisfying the equation.
\end{enumerate}

\end{enumerate}


\vfill
\begin{center}
\begin{BVerbatim}

\end{BVerbatim}
\end{center}

\end{document}
