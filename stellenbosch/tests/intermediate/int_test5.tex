\documentclass{article}

\usepackage{verbatim}
\usepackage{amsmath,amsfonts}
\usepackage{enumerate}
\usepackage{fancyvrb}
\usepackage{graphicx}
% \usepackage{fullpage}

\newcommand{\ds}{\ensuremath{\displaystyle}}
\newcommand{\floor}[1]{\ensuremath{\left\lfloor#1\right\rfloor}}
\newcommand{\fracpart}[1]{\ensuremath{\left\{#1\right\}}}

\title{Intermediate Test 5}
\author{Stellenbosch Camp 2018}
\date{Time: $4$ hours}


\begin{document} \maketitle

\begin{enumerate}[1.]

\item % 102 Combinatorial Problems #2
The student lockers at Olympic High are numbered consecutively beginning with locker number 1. The plastic digits used to number the lockers cost $3$ cents per piece. Thus, it costs $3$ cents to number locker $9$ and $6$ cents to number locker $42$. If it costs R206.91 to label all the lockers, how many lockers are there at the school?


\vspace{6pt}
\item % UK IMO 2001 IMO training problem sheet
Given the equation $x^{2018} = y^x$,
\begin{enumerate}
  \item find all pairs $(x,y)$ of solutions with $x$ prime and $y$ a positive integer;
  \item find all pairs $(x,y)$ of positive integers satisfying the equation.
\end{enumerate}


\vspace{6pt}
\item % Phil's Brazilian Inequalities
Prove that for any three positive real numbers $a$, $b$ and $c$, \[ a^4 +b^4 +c^4 \geq a^2bc +b^2ca +c^2ab. \]


\item % Thanks to Andrew
Consider two circles $\Gamma_1$ and $\Gamma_2$ that intersect at points A and B. Let $l$ be a line tangent to circles $\Gamma_1$ and $\Gamma_2$ at $S$ and $T$, respectively. Lines $AB$ and $ST$ intersect at point $M$. Furthermore line $BT$ intersect circle $\Gamma_1$ again at point $R$. Let the intersection of $MR$ and $SB$ be $X$ and the intersection of $TX$ and $RS$ be $C$. 
Prove that $CB$ and $ST$ are parallel.  


\vspace{6pt}
\item % 102 Combinatorial Problems #11
Determine the number of ways to choose five numbers from the first eighteen positive integers such that any two chosen numbers differ by at least $2$.


\vspace{6pt}
\item % NT Sheet
Show that there exists an infinite arithmetic progression of natural numbers such that the first term is $16$ and the number of positive divisors of each term is divisible by $5$. Of all such sequences, find the one with the smallest possible positive common difference.


\end{enumerate}


\vfill
\centering
\begin{BVerbatim}
                             .-.
(___________________________()6 `-,
(   ______________________   /''"`
//\\                      //\\
"" ""                     "" ""
\end{BVerbatim}

\end{document}
