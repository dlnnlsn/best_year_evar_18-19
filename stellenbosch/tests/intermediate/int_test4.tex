\documentclass{article}

\usepackage{verbatim}
\usepackage{amsmath,amsfonts}
\usepackage{enumerate}
\usepackage{fancyvrb}

\newcommand{\ds}{\ensuremath{\displaystyle}}
\newcommand{\floor}[1]{\ensuremath{\left\lfloor#1\right\rfloor}}
\newcommand{\fracpart}[1]{\ensuremath{\left\{#1\right\}}}

\title{Intermediate Test 4}
\author{Stellenbosch Camp 2018}
\date{Time: $2\frac{1}{2}$ hours}


\begin{document}

\maketitle

\begin{enumerate}[1.]

\item % The Phil
How many numbers from $1$ to $2018$ inclusive can be written as the difference of two perfect squares?


\vspace{12pt}
\item %


\vspace{12pt}
\item %


\vspace{12pt}
\item % AoPS:  https://artofproblemsolving.com/community/c6t309f6h455752_integers_in_infinite_chessboard
Prove that it is impossible to write a positive integer in every cell of an infinite chessboard, in such a manner that, for all positive integers $m, n$, the sum of numbers in every $m\times n$ rectangle is divisible by $m + n$.


\vspace{12pt}
\item % Asiatic Pacific Maths Olympiad, 1989, Q3
Let $A_1, A_2, A_3$ be three points in the plane, and for convenience, let $A_4 = A_1$, $A_5 = A_2$. For $n = 1, 2$ and $3$, suppose that $B_n$ is the midpoint of $A_n A_{n+1}$ and suppose that $C_n$ is the midpoint of $A_n B_n$. Suppose that $A_n C_{n+1}$ and $B_n A_{n+2}$ meet at $D_n$ and that $A_n B_{n+1}$ and $C_n A_{n+2}$ meet at $E_n$. Calculate the ratio of the area of triangle $\triangle D_1 D_2 D_3$ to the area of triangle $\triangle E_1 E_2 E_3$.


\end{enumerate}


\vfill
\centering
\begin{BVerbatim}
             .--~~,__
:-....,-------`~~'._.'
 `-,,,  ,_      ;'~U'
  _,-' ,'`-__; '--.
 (_/'~~      ''''(;
\end{BVerbatim}

\end{document}
