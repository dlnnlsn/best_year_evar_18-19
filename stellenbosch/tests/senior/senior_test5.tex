\documentclass[a4paper, 12pt]{article}
\usepackage[margin=2cm]{geometry} 
\usepackage{amsmath,amsfonts}
\usepackage{amssymb,amsthm}
\usepackage{tikz,tkz-euclide}
\usepackage[cm]{fullpage}
\usepackage{fancyvrb}

\usetikzlibrary{calc,patterns,angles,quotes}
\usetkzobj{all}

\title{Senior Test 5}
\author{Stellenbosch Camp 2018}
\date{Time: $4$ hours}

\begin{document} \maketitle

\begin{enumerate}

% 102 Combinatorial Problem #11
\item[1.] Determine the number of ways to choose five numbers from the first eighteen positive integers such that any two chosen numbers differ by at least $2$.


\vspace{6pt}

% NT Sheet
\item[2.]  Show that there exists an infinite arithmetic progression of natural numbers such that the first term is $16$ and the number of positive divisors of each term is divisible by $5$. Of all such sequences, find the one with the smallest possible positive common difference.


\vspace{6pt}
% Lauri Hallila
\item[3.]      Suppose that Romeo and Juliet each have a regular tetrahedron, to the vertices of which some positive real numbers are assigned. They associate, to each edge of their tetrahedra, the product of the two numbers assigned to its end points. Then they write on each face of their tetrahedra the sum of the three numbers associated to its three edges. The four numbers written on the faces of Romeo’s tetrahedron turn out to coincide with the four numbers written on Juliet’s tetrahedron. Does it follow that the four numbers assigned to the vertices of Romeo’s tetrahedron are identical to the four numbers assigned to the vertices of Juliet’s tetrahedron?


\vspace{6pt}

% Source: https://artofproblemsolving.com/community/c6t325f6h79786_determine_pairs_ab
% Original: APMO 1999
\item[4.] Determine all pairs $(a,b)$ of integers with the property that the numbers $a^2+4b$ and $b^2+4a$ are both perfect squares.


\vspace{6pt}

% USAMO 1999 (Problem B2 adjusted)
\item[5.]   Two players play a game on a line of 2018 squares. Each player in turn puts either S or O into an empty square. The game stops when three adjacent squares contain S, O, S in that order and the last player wins. If all the squares are filled without getting S, O, S, then the game is drawn. Show that the second player can always win.


% Bulgaria MO 2011
\item[6.]
Isosceles $\triangle ABC$ with $AC=BC$ is inscribed in a circle $k$. A point $M$ lies on the side $BC$. A point $N$ from the ray $AM$ ($M$ lies between $A$ and $N$) is such that $AN=AC$. The circumcircle of $\triangle MCN$ intersects $k$ at $C$ and $P$, where $P$ is on the arc $BC$ not containing $A$. The lines $AB$ and $CP$ intersect at $Q$. Prove that $\angle QMB = \angle QMN$.


\end{enumerate}

\vfill

\centering
\begin{BVerbatim}
     |\_/|                  
     | @ @   Woof! 
     |   <>              _  
     |  _/\------____ ((| |))
     |               `--' |   
 ____|_       ___|   |___.' 
/_/_____/____/_______|
\end{BVerbatim}

\vspace{12mm}


\end{document}
