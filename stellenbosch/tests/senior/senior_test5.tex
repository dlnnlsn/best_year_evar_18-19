\documentclass[a4paper, 12pt]{article}
\usepackage[margin=2cm]{geometry} 
\usepackage{amsmath,amsfonts}
\usepackage{amssymb,amsthm}
\usepackage{tikz,tkz-euclide}
\usepackage{fullpage}
\usepackage{fancyvrb}

\usetikzlibrary{calc,patterns,angles,quotes}
\usetkzobj{all}

\title{Senior Test 5}
\author{Stellenbosch Camp 2018}
\date{Time: $4$ hours}


\begin{document} \maketitle

\begin{enumerate}

% 102 Combinatorial Problem #11
\item[1.] Determine the number of ways to choose five numbers from the first eighteen positive integers such that any two chosen numbers differ by at least $2$.


\vspace{6pt}

% 
\item[2.] 


\vspace{6pt}

% 
\item[3.] 


\vspace{6pt}

% Source: https://math.stackexchange.com/questions/1682542/2009-benelux-math-olympiad-bxmo-number-theory-problem
% Original: 2009 Benelux Math Olympiad (BxMO)
\item[4.] Let $n$ be a positive integer and let $k$ be an odd positive integer. Moreover, let $a$, $b$ and $c$ be integers (not necessarily positive) satisfying the equation
\begin{equation*}
    a^n + kb = b^n + kc = c^n + ka
\end{equation*}

Prove that $a=b=c$.


\vspace{6pt}

% USAMO 1999 (Problem B2 adjusted)
\item[5.]   Two players play a game on a line of 2018 squares. Each player in turn puts either S or O into an empty square. The game stops when three adjacent squares contain S, O, S in that order and the last player wins. If all the squares are filled without getting S, O, S, then the game is drawn. Show that the second player can always win.


\end{enumerate}

\vfill

\centering
\begin{BVerbatim}
\end{BVerbatim}


\end{document}
