\documentclass[a4paper, 12pt]{article}
\usepackage[margin=2cm]{geometry} 
\usepackage{amsmath,amsfonts}
\usepackage[cm]{fullpage}
\usepackage{fancyvrb}

\title{Senior Test 1}
\author{Stellenbosch Camp 2018}
\date{Time: $2 \frac{1}{2}$ hours}

\begin{document} \maketitle

\begin{enumerate}


\item[1.] Prove that $m + n \leq \textrm{gcd}(m, n) + \textrm{lcm}(m, n)$ for all positive integers $m, n$. When does equality occur?

% AoPS
\item[2.] Find all functions $f : \mathbb{R} \to \mathbb{R}$ such that
$$ f(f(x+y)) = x + f(y) $$
for all $x, y \in \mathbb{R}$.

% Andrew: 
\item[3.] Let $ABC$ be an acute angled triangle. The circle with diameter $AB$ intersects the altitude from $C$ at $M$ and $N$. The circle with diameter $AC$ intersects the altitude from $B$ at $P$ and $Q$. Prove $M$, $N$, $P$, and $Q$ all lie on a common circle.


%An acute-angled triangle $ABC$ is given in the plane. The circle with diameter $AB$ intersects altitude $CC'$ at its extension at points $M$ and $N$, and the circle with diameter $AC$ intersects altitude $BB'$ at its extension at $P$ and $Q$. Prove that the points $M, N, P, Q$ lie on a common circle.


\item[4.] Let $n$ be a positive integer and define $S_n = \{1, 2, 3, \dots, n\}$. We denote a non-empty subset $T$ of $S_n$ as \textit{balanced} if the median of $T$ is equal to the average of $T$. For each $n \geq 1$, prove that the number of balanced subsets of $S_n$ is odd. 

The median of a subset $T$ is defined as follows: Let $T = \{a_1, a_2, \dots, a_k\}$ with the elements listed in increasing order $a_1 < a_2 < \dots < a_k$. If $k$ is odd, then the median of $T$ is the element $a_{(k+1)/2}$. Otherwise, if $k$ is even, then the median is the average of $a_{k/2}$ and $a_{k/2 + 1}$.

\item[5.] Set $S = \{1, 2, 3, ..., 2018\}$. Let $n$ be a positive integer such that, among any $n$ pairwise coprime numbers in $S$, there exists at least a prime number. Find the minimum value of $n$.


\end{enumerate}

\vfill

\centering
\begin{BVerbatim}
       /^-^\
      / o o \
     /   Y   \
     V \ v / V
       / - \
      /    |
(    /     |
 ===/___) ||
\end{BVerbatim}

\vspace{10mm}

\end{document}
