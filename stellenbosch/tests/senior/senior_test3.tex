\documentclass[a4paper, 12pt]{article}
\usepackage[margin=2cm]{geometry} 
\usepackage{amsmath,amsfonts}
\usepackage[cm]{fullpage}
\usepackage{fancyvrb}

\title{Senior Test 3}
\author{Stellenbosch Camp 2018}
\date{Time: $2 \frac{1}{2}$ hours}

\begin{document} \maketitle

\begin{enumerate}

% ($2nd Asiatic pacific mathematical olympiad , 1990)
\item[1.] Consider all the triangles $ABC$ which have a fixed base $AB$ and whose altitude from $C$ is a constant $h$. For which of these triangles is the product of its altitudes a maximum? 

% 
\item[2.]  
\\


% 
\item[3.]  \\ Consider an 8x8 grid of squares. On each square is a lightbulb initially off. A move consists of choosing a square and either the verticle or horizonal direction, and toggling the lightbuld on that square and it's immediate neighbours, in the choosen direction. For clarity this means that usually 3 bulbs are flipped unless the square is on an edge in which case 2 bulbs may be flipped. After some amount of moves a single bulb is on (the other 63 are off). Determine which of the 64 bulbs can possibly be on. 



% 
\item[4.]   \\

% 
\item[5.]   \\


\end{enumerate}

\vfill

%Razza - a mix of Lab and Chow
\centering
\begin{BVerbatim}
   / \__
  (    @\___
  /         O
 /   (_____/
/_____/   U
\end{BVerbatim}

\vspace{12mm}

\end{document}
