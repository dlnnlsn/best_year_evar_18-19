\documentclass[a4paper, 12pt]{article}
\usepackage{amsmath,amsfonts}
\usepackage{fancyvrb}

\title{Senior Test 3}
\author{Stellenbosch Camp 2018}
\date{Time: $2 \frac{1}{2}$ hours}

\begin{document} \maketitle

\begin{enumerate}

% ($2nd Asiatic pacific mathematical olympiad , 1990)
\item[1.] Consider all the triangles $ABC$ which have a fixed base $AB$ and whose altitude from $C$ is a constant $h$. For which of these triangles is the product of its altitudes a maximum? 


\vspace{12pt}
% 
\item[2.]  



\vspace{12pt}
% (Iberoamerican Mathematical Olympiad shortlist 2009)
\item[3.]
Consider an 8x8 grid of squares. On each square is a lightbulb initially off. A move consists of choosing a square and either the verticle or horizonal direction, and toggling the lightbuld on that square and it's immediate neighbours, in the chosen direction. For clarity this means that usually 3 bulbs are flipped unless the square is on an edge in which case 2 bulbs may be flipped. After some amount of moves a single bulb is switched on (the other 63 are off). Determine which of the 64 bulbs can possibly be on. 


\vspace{12pt}
% 2017/2018 British MO Round 2
\item[4.]
It is well known that, for each positive integer $n$, \[ 1^3 +2^3 +\dotsb +n^3 = \frac{n^2(n+1)^2}{4} \] and hence is a square. Determine whether or not there is a positive integer $m$ such that \[ (m+1)^3 +(m+2)^3 +\dotsb +(2m)^3 \] is a square.


\vspace{12pt}
% 
\item[5.]


\end{enumerate}

\vfill

%Razza - a mix of Lab and Chow
\centering
\begin{BVerbatim}
   / \__
  (    @\___
  /         O
 /   (_____/
/_____/   U
\end{BVerbatim}

\vspace{12mm}

\end{document}
