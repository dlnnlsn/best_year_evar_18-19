\documentclass{article}
\usepackage[margin=0.8in]{geometry}
\usepackage{amsmath,amssymb,amsthm}
\usepackage{enumerate}

\newtheorem*{theorem*}{Theorem}
\newtheorem{theorem}{Theorem}
\newtheorem*{definition}{Definition}
\newtheorem*{notate}{Notation}


\newcommand\ora[1]{\overrightarrow{#1}}


\title{Stellenbosch Camp 2018: Senior Geometry\\ \centering Lecture 2: Inversion Basics}
\date{}
\begin{document}
	\maketitle
	\hrulefill
	\begin{definition}[Circle Inversion]
		Let $\Omega$ be a circle with circumcentre $O$ and radius $r$. We say that $T$ is an inversion in the circle $\Omega$ if for any point $A \neq O$ in the plane the following is true:
		\begin{enumerate}
			\item $T$ is a map that sends $A$ to a point $A'$
			\item $A'$ is on the ray $OA$
			\item $OA \cdot OA' = r^{2}$
		\end{enumerate}
		We say that $A'$ is the image of $A$.
	\end{definition}
	\begin{notate}
		When referencing a circle inversion called $T$, in a circle $\Omega$, with centre $O$ and radius $r$. There are a few accepted notations for short:
		\begin{itemize}
			\item $T(\Omega)$
			\item $T(O,r)$
		\end{itemize}
	\end{notate}
	\begin{theorem}
		If $A'$ is the image of $A$ with respect to a circle $\Omega$, then $A$ is the image of $A'$ with respect to $\Omega$.
	\end{theorem}
	\begin{theorem}
		Let $A$ be a point on the circumference of a circle $\Omega$. Then $A$ is the image of $A$ with respect to the circle $\Omega$.
	\end{theorem}
	\begin{theorem}
		Let $\Omega$ be a circle, and let $A$,$B$ be 2 points. Let $A'$ and $B'$ be the images of $A$ and $B$, with respect to $\Omega$, respectively. Then $A$, $A'$, $B$, and $B'$ all lie on a common circle.
	\end{theorem}
	\begin{definition}[Inverting a Set of Points]
		Let $S$ be a set of points (E.g. a line, a circle, etc). Then the image of $S$ with respect to a circle $\Omega$, denoted $S'$, is the set containing the image of every point in $S$.\\
		In symbols:
		$$ S' := \{ P' \in  \ora{OP} : OP \cdot OP' = r^{2},  \forall P \in S  \} $$
	\end{definition}
	\begin{theorem}
		Let $\Omega$ be a circle with circumcentre $O$, and let $\ell$ be a line not passing through $O$. Then the image of $\ell$ with respect to $\Omega$ is a circle passing through $O$.\\
		Conversely, if $\Gamma$ is a circle passing through $O$, then the image of $\Gamma$ with respect to $\Omega$ is a line not passing through $O$.\\
	\end{theorem}
	\begin{theorem}
		Let $\Omega$ be a circle with circumcentre $O$, and let $\Gamma$ be a circle not passing through $O$. Then the image of $\Gamma$ with respect to $\Omega$ is also a circle not passing through $O$.
	\end{theorem}
	\begin{theorem}
		Let $\Omega$ be a circle with circumcentre $O$, and let $\ell$ be a line passing through $O$. Then the image of $\ell$ with respect to $\Omega$ is itself.
	\end{theorem}
	\begin{theorem}
		Let $\Omega$ be a circle with circumcentre $O$, and let $\Gamma$ be a circle that is orthogonal to $\Omega$ (intersecting at right angles). Then the image of $\Gamma$ with respect to $\Omega$ is itself.
	\end{theorem}
	\begin{theorem}
		Let $T(O,r)$ be an inversion, and let $\alpha$ and $\beta$ be 2 circles in the plane. Then the angles between the intersections of of the circles is invariant under $T$.
	\end{theorem}
\end{document}