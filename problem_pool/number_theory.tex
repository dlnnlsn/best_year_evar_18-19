\documentclass[a4paper, 12pt]{article}
 
\usepackage[margin=2cm]{geometry} 
%\usepackage[framed,numbered,autolinebreaks,useliterate]{mcode}
\usepackage{amsmath,amsthm,amssymb}
\usepackage{verbatim}
\usepackage{graphicx}
\usepackage{listings}
\usepackage{longtable}
\usepackage{array}
\usepackage{float}
\usepackage{minted}
\usepackage[utf8]{inputenc}
\usepackage[english]{babel}
\usepackage[usenames, dvipsnames]{color}
\usepackage[dvipsnames]{xcolor}

\setlength\parindent{0pt} % Removes all indentation from paragraphs
%\pagenumbering{gobble} % No page numbers

% Make solutions visible/invisible
\newcommand{\solution}[1]{%
   \emph{Solution}: 
    #1            
}       
% Comment/Uncomment the following line to toggle visibility of solutions:
\renewcommand{\solution}[1]{}
 

\title{2018 Stellenbosch Mathematics Camp \\ Senior Number Theory Problem Set}
\author{Robin Visser}
\date{December 2018}

\begin{document}

\maketitle




\section{Diophantine equations}

\begin{enumerate}


% Source: https://artofproblemsolving.com/community/c6t325f6h86541_easy_diophantine_equation
% Original: AMO 2006
\item Solve for positive integers $m, n$:
\begin{equation*}
    1 + 5 \cdot 2^m = n^2
\end{equation*}

\solution{ We have $(n-1)(n+1) = 5 \cdot 2^m$. Note that $n$ is odd, which implies both $n-1$ and $n+1$ are even. Furthermore, gcd($n-1, n+1$) = 2. Therefore, we have either $v_2(n-1) = 1$ or $v_2(n+1) = 1$. Noting the cases for 5, this yields the following possibilities:
\begin{align*}
    n - 1 &= 2 \quad \textrm{or} \quad n - 1 = 10 \\
    \textrm{or} \quad  n + 1 &= 2 \quad \textrm{or} \quad n + 1 = 10 
\end{align*}
Noting all cases, one obtains that only $n=9$ yields a valid solution where $m = 4$. Thus the only solution is $(n, m) = (9, 4)$. \qed

}

% Source: https://artofproblemsolving.com/community/c6t31442f6h1533077_romanian_district_olympiad_grade_viii
\item Solve for $m, n \in \mathbb{Z}$:
\begin{equation*}
    9m^2+3n=n^2+8
\end{equation*}

\solution{ We consider the equation as a quadratic in $n$
\begin{equation*}
    n^2-3n+8-9m^2=0
\end{equation*}
Solving for the discriminant $\Delta$ yields
\begin{equation*}
    \Delta = 9 - 4(8 - 9m^2) = 36m^2 - 23
\end{equation*}
For $n$ to be an integer, we require $\Delta$ to be a perfect square. Thus $36m^2 - 23 = l^2$ for some $l \geq 0$. Factoring, we obtain $(6m+l)(6m-l)=23$. As 23 is prime, this yields
\begin{equation*}
    6m \mp l = \pm 1 \quad \textrm{and} \quad 6m \pm l = \pm 23
\end{equation*}
Solving this system yields $m - 2$ and $l = 11$, or $m = -2$ and $l = 11$. This therefore gives the following four solutions: $(m, n) \in \{(2, -4), (2, 7), (-2, -4), (-2, 7) \}$. \qed
}


% Source: https://artofproblemsolving.com/community/c6t31442f6h1577363_nt_diophantine
\item Solve for $a, d \in \mathbb{Z}$:
 \begin{equation*}
     a^4 + 120a = d^2
 \end{equation*}
 
 \solution{Note if $a = 0$, then $d = 0$. Assuming now that $a > 0$, we have $a^4 < d^2$. Thus $(a^2 + 1)^2 \leq d^2 = a^4 + 120a$. This yields the following inequality for $a$:
 \begin{equation*}
     2a^2 - 120a + 1 \leq 0
 \end{equation*}
 This implies $0 < a < 60$. A very tedious way to conclude, is to check all remaining cases. An alternate approach is to bound $d^2$ by a weaker interval, such as noting that $a^4 < d^2 < (a^2 +6)^2 $ for $a \geq 10$. Then we only need to check the cases $0 < a < 10$, otherwise assuming $a \geq 10$, we have either $d = a^2 + 2$ or $d = a^2 + 4$, noting that the parity of $a$ and $d$ are the same. \\
 
 Doing this approach, we note that $a \in \{1, 2, 3, 5\}$ yields valid solutions for $0 < a < 10$. Otherwise, substituting $d = a^2 + 2$ or $d = a^2 + 4$ into the given equation yields no further solutions.
 
 Doing the same approach for negative integers $a$, we obtain 
 $$(a, d) \in \{(0, 0), (1, \pm 11), (2, \pm 16), (3, \pm 21), (5, \pm 35), (-5, \pm 5), (-6, \pm 24), (8, \pm 56) \}.$$
 \qed
 
 \textit{Remark:}  The factor 120 can be made smaller to reduce case checking. One can also restrict to strictly positive integer solutions.
 }



\item Solve for $x, y \in \mathbb{Z}$:
\begin{equation*}
    7^x + x^4 + 47 = y^2
\end{equation*}

\solution{ First consider the equation mod 4. This yields $(-1)^x + x^4 - 1 \equiv y^2$ (mod 4). Assuming that $x$ is odd, this furthermore gives $x^4 - 2 \equiv y^2$ (mod 4). However, as squares (and thus fourth powers) are only 0 or 1 (mod 4), this gives a contradiction.

Therefore $x$ is even. Let $x = 2a$. This gives $7^{2a} + 16a^4 + 47 = y^2$. Note that $(7^a)^2 < y^2$, thus $(7^a + 1)^2 \leq y^2$, which yields
\begin{align*}
    7^{2a} + 2 \cdot 7^a + 1 &\leq y^2 = 7^{2a} + 16a^4 + 47 \\
    \implies 7^a &\leq 8a^4 + 23
\end{align*}
Now, one can easily prove by induction that this is false if $a \leq 4$. Thus $a \in \{1, 2, 3\}$. Checking each case yields the only solution is $a = 2$, thus we have $(x, y) = (4, 52)$ as the only solution. \qed
}


% Source: https://artofproblemsolving.com/community/c6t325f6h79786_determine_pairs_ab
% Original: APMO 1999
\item Determine all pairs $(a,b)$ of integers with the property that the numbers $a^2+4b$ and $b^2+4a$ are both perfect squares.

\solution{ If either $a$ or $b$ is 0, then both must be perfect squares. If $a = b$, then we must solve $a^2 + 4a = k^2$ where $k \in \mathbb{Z}$. This gives us $(a+2-k)(a+2+k) = 4$ which can easily be solved to give the solution $a = b = -4$. If $a = -b$, then we similarly obtain $(a-2-k)(a-2+k) = 4$ which yields no solutions. \\

We can now assume $|a| > |b| \geq 1$.  If $|a| > 4$, we have
\begin{equation*}
    (|a| - 4)^2 = a^2 - 8|a| + 16 \leq a^2 - 4|a| \leq a^2 + 4b \leq a^2 + 4|a| < (|a| + 2)^2
\end{equation*}
This bounds $a^2 + 4b$ between two squares. Furthermore, noting that $a^2 + 4b$ and $b^2 + 4a$ have the same pairty, we thus have that either $a^2 + 4b = a^2$ or $a^2 + 4b = (|a| - 2)^2$. The former case yields $b = 0$ which has been covered, thus we assume $a^2 + 4b = (|a| - 2)^2$, which implies $b = 1 - |a|$.

Therefore, $b^2 + 4a = a^2 - 2|a| + 1 + 4a$ is a perfect square. We consider two cases:

\textbf{Case 1}: $a > 0$. Thus $b^2 + 4a = a^2 + 2a + 1 = (a+1)^2$ which is a square.

\textbf{Case 2}: $a < 0$. Thus $b^2 + 4a = a^2 + 6a + 1 = (a + 3)^2 - 8$. Now, the only case which will give two squares a difference of 8 apart is $1, 9$, which yields $a = -6$ and $b = -5$.

Finally, we consider the cases where $|a| \leq 4$. This only yields the solution $(-4, -4)$. We thus conclude the solutions are
\begin{align*}
    a = 0 &\textrm{ and } b = k^2, \quad k \geq 0 \\
    \textrm{ or } \quad a = k^2 &\textrm{ and } b = 0, \quad k \geq 0 \\
    \textrm{ or } \quad a = k &\textrm{ and } b = 1-k, \quad k \in \mathbb{Z} \\
    \textrm{ or } \quad (a, b) &\in \{(-4, -4), (-5, -6), (-6, -5) \}
\end{align*}
\qed

}



% Source: https://artofproblemsolving.com/community/c6h1748125_v_p_will_be_useful
% (refers to https://artofproblemsolving.com/community/c6h1135658p5301680 )
\item Find all positive integers $a, b$ such that $a! + b! = a^b + b^a$.

\solution{ We first consider $a = b$. Thus $a! = a^a$ which only has the solution $a = 1$. We now assume wlog that $a > b$. We consider two cases:

\textbf{Case 1:} gcd($a, b) > 1$. Let $p$ be prime such that $p \mid \textrm{gcd}(a, b)$ and let $a = pm$ and $b = pn$. Thus
\begin{equation*}
    (pm)!+(pn)!=(pm)^{pn}+(pn)^{pm}
\end{equation*}
Noting the highest power of $p$ dividing the equation, we have
\begin{equation*}
    v_p(LHS) = v_p((pm)!+(pn)!) = v_p((pn)!) = \sum_{k=1}^\infty \left\lfloor \frac{pn}{p^k} \right\rfloor < \sum_{k=1}^\infty \frac{pn}{p^k} = \frac{pn}{p-1} \leq pn \leq v_p(RHS)
\end{equation*}
which gives a contradiction. \\

\textbf{Case 2:} gcd($a, b) = 1$. We define $k = b - a$, where $k > 0$ and gcd$(b, k) = 1$. Therefore
\begin{equation*}
    (b+k)! + b! = (b+k)^b + b^{b+k}
\end{equation*}
Now clearly $b \mid ((b+k)! + b!)$, thus $b \mid (b+k)^b$. By the binomial theorem, this implies $b \mid k^b$. However, noting that $b, k$ are coprime, this yields $b = 1$. Thus $a! + 1 = a + 1$, which implies $a = 2$ (as $a > b = 1$).

Therefore we have the solutions $(a, b) \in \{(1, 1), (1, 2), (2, 1)\}$. \qed
}



% Source: https://artofproblemsolving.com/community/c6t31442f6h1514449_equation_in_z
\item Solve for $x, y, z \in \mathbb{Z}$:
 
 \begin{equation*}
      2(x^2+y^2+z^2)=(x-y)^3+(y-z)^3+(z-x)^3
 \end{equation*}
 
\solution{ Consider the left hand side $(x-y)^3+(y-z)^3+(z-x)^3$. Note that this evaluates to 0 if $x = y$ or $y = z$ or $z = x$. Thus $(x-y)$, $(y-z)$ and $(z-x)$ are all factors. (one can check this explicitly as well). Quotienting out these factors from the LHS yields a quotient of 3. We thus obtain
 \begin{equation*}
     2(x^2+y^2+z^2)=3(x-y)(y-z)(z-x)
 \end{equation*}
 Todo conclude.  Solutions, up to cyclic order and sign are 
\begin{align*}
    \{&(0, 0, 0), (10, 20, 25), (30, 40, 70), (65, 70, 125), (91, 104, 143), \\
    &(105, 150, 165), (115, 130, 175), (140, 259, 266) \dots \}
\end{align*}.

 }



\end{enumerate}


\section{Divisibility}

\begin{enumerate}

% Source: Somewhere on Quora
\item  Prove that $n$ does not divide $2^n - 1$ for all positive integers $n > 1$.

% Note: It is NOT true that n and 2^n - 1 are always coprime

\solution{ Assume for contradiction that $n > 1$ such that $n$ divides $2^n - 1$. Clearly $2^n - 1$ is always odd, thus $n$ must be odd. Let $p$ be the smallest prime dividing $n$. Thus $p$ divides $2^n - 1$. Therefore
$$ 2^n \equiv 1 (\textrm{mod } p) $$
However, by Fermat's Little, we have
$$ 2^{p-1} \equiv 1 (\textrm{mod } p) $$
Thus, the order of 2 (mod $p$) must divide both $n$ and $p-1$. However, as $p$ is the smallest odd prime diving $n$, we have $n$ and $p-1$ are coprime (noting $n$ is odd). Thus the order of 2 (mod $p$) is 1, which is a contradiction. \qed \\

\textit{Remark:} Perhaps this can be generalised to determine the $n$ which is coprime to $2^n-1$.
}


% Source: https://artofproblemsolving.com/community/c6t328f6h3701_divisibility_impossibility
% Andrei Jorza's Number Theory handout
\item  Prove that $n$ does not divide $2^{n-1} + 1$ for all positive integers $n > 1$.

\solution{ Doing a contradiction argument, let $p$ be a prime divisor of $n$. Thus $2^{n-1} \equiv -1$ (mod $p$) which implies $2^{2(n-1)} \equiv 1$ (mod $p$). Let $t$ denote the order of $2$ modulo p. Thus $t \mid 2(n-1)$ and $t \not \mid (n-1)$. Let $n-1 = 2^k u$ where $u$ is odd. We thus have $2^{k+1} \mid t$. Now, by Fermat's Little, $t \mid p-1$. We thus have $2^{k+1} \mid p-1$, which implies $p \equiv 1$ (mod $2^{k+1}$) for all prime divisors $p$ of $n$. Thus $n \equiv 1$ (mod $2^{k+1}$), and thus $2^{k+1} \mid n-1$, which clearly contradicts $u$ being odd. \qed

}

% https://artofproblemsolving.com/community/c6t177f6h1744223_prove_that
\item Let $p$ be an odd prime. Prove that $(p-1)^p + 1$ is divisible by $p^2$, but not divisible by $p^3$.

\solution{ We use the binomial theorem. Note
\begin{align*}
    (p-1)^p + 1 &= \sum_{k=0}^\infty \binom{p}{k} p^k (-1)^{p-k} + 1 \\
    &= \sum_{k=2}^\infty \binom{p}{k} p^k (-1)^{p-k} + \binom{p}{1} p^1 (-1)^{p-1} + + \binom{p}{0} p^0 (-1)^{p-0}    + 1 \\
    &= \sum_{k=2}^\infty \binom{p}{k} p^k (-1)^{p-k} + p^2
\end{align*}
noting that $p$ is odd, thus $(-1)^p = -1$. We also note that $\binom{p}{k} p^k$ is divisible by $p^3$ for all $k \geq 2$. Thus
\begin{equation*}
    (p-1)^p + 1 = \sum_{k=2}^\infty \binom{p}{k} p^k (-1)^{p-k} + p^2 \equiv p^2 \quad  (\textrm{mod } p^3)
\end{equation*}
which proves the claim. \qed
}

% Source: https://artofproblemsolving.com/community/c6t325f6h376734_divisiblity_with_a_binomial
% Original: Italy TST 2002
\item Prove that for each prime number $p$ and positive integer $n$, $p^n$ divides
\begin{equation*}
    \binom{p^n}{p} - p^{n-1}
\end{equation*}

\solution{ The case $p = 2$ can easily be verified. Now, assume $p$ is an odd prime. Note
\begin{equation*}
    \binom{p^n}{p} - p^{n-1} = \frac{(p^n)!}{p! (p^n - p)!} - p^{n-1} = p^{n-1} \frac{(p^n - 1)!}{(p-1)! (p^n - p)!} - p^{n-1} = p^{n-1} \left( \binom{p^n - 1}{p-1} - 1  \right)
\end{equation*}
Thus $p^n$ divides $\binom{p^n}{p} - p^{n-1}$ iff $p$ divides $\binom{p^n - 1}{p-1} - 1$. Noting that $(p-1)!$ is coprime to $p$, we have
\begin{align*}
    \binom{p^n - 1}{p-1} \equiv 1 \quad (\textrm{mod } p) \\
    \iff \quad (p^n - 1)(p^n - 2) \dots (p^n - p + 1) \equiv 1 \cdot 2 \dots (p-1) \quad (\textrm{mod } p) 
\end{align*}
Reducing modulo $p$, we obtain $(-1)^{p-1} (p-1)! \equiv (p-1)!$ (mod p) which is clearly true as $p$ is odd. \qed
}



% https://artofproblemsolving.com/community/c6t328f6h1755_abc13
\item If positive integers $a,b,c$ are such that $b$ divides $a^3$, $c$ divides $b^3$, and $a$ divides $c^3$, prove that $abc$ divides $(a+b+c)^{13}$.

\solution{ We easily note that $a, b, c$ must all have the same prime factors. Let $p$ be a prime which divides $a$ (and thus $b$ and $c$). We note
\begin{align*}
    v_p((a+b+c)^{13}) = 13 v_p(a+b+c) &\geq 13 \textrm{min}\{v_p(a), v_p(b), v_p(c)\} \\
    &= (3^2 + 3 + 1) \textrm{min}\{v_p(a), v_p(b), v_p(c)\} \\
    &\geq v_p(a) + v_p(b) + v_p(c) = v_p(abc)
\end{align*}
Noting this for each prime factor $p$, we obtain that $abc$ divides $(a+b+c)^{13}$. \qed

\textit{Remark:}  Indeed, one can prove more generally that if $b$ divides $a^n$, $c$ divides $b^n$ and $a$ divides $c^n$, then $abc$ divides $(a + b + c)^{n^2 + n + 1}$.

}


\end{enumerate}

\section{Sequences}

\begin{enumerate}
    
% Source: https://artofproblemsolving.com/community/c6t177f6h1635125_sequence_is_all_prime
% Originally from All-Russia 2018 Grade 9 P1
\item Suppose $a_1,a_2, \dots$ is an infinite strictly increasing sequence of positive integers and $p_1, p_2, \dots$ is a sequence of distinct primes such that $p_n \mid a_n$ for all $n \geq 1$ and such that $a_n - a_k = p_n - p_k$ for all $n,k \geq 1$. Prove that the sequence $(a_n)$ consists only of prime numbers.

\solution{ Let $c = a_1 - p_1$, and assume for contradiction that $c \geq 1$. For all $n$, we have $a_n - p_n = a_1 - p_1 = c$. Now, as $p_n \mid a_n$, we have $p_n \mid (a_n - p_n) = c$, thus $p_n \leq c$. Hence $a_n = c + p_n \leq 2c$. This implies the sequence $(a_n)$ is bounded which is a contradiction. Thus $c = 0$, which implies $a_n = p_n$ for all $n$. \qed
}


% Source: https://artofproblemsolving.com/community/c6t325f6h416872_smallest_common_difference_of_arithmetic_progressions
% Original:  Turkey IMO TST 1993 #1
\item Show that there exists an infinite arithmetic progression of natural numbers such that the first term is $16$ and the number of positive divisors of each term is divisible by $5$. Of all such sequences, find the one with the smallest possible positive common difference.

\solution{ Note that the sequence given by $a_k = 16 + 32k$ satisfies the problem condition, as $16 + 32k = 2^4 (2k + 1)$, thus $v_2(a_k) = 4$, which proves $4 + 1 = 5$ divides the number of divisors for all $k$.

Secondly, we note that the smallest possible common difference is 32, as the next smallest number after 16 that has number of divisors a multiple of 5 is $48 = 2^4 \cdot 3$. (indeed, if $n = p_1^{a_1} \dots p_k^{a_k}$, then $d(n) = (a_1 +1)\dots (a_k +1)$, which implies $5 \mid (a_i + 1)$ for some $i$. Thus, $a_i \geq 4$ for some $i$). \qed
}


% Source: https://artofproblemsolving.com/community/c6h2461
% (referred to in https://artofproblemsolving.com/community/c6t328f6h110917_a_sequence_problem)
\item Let $k$ be a positive integer and let $a_1, a_2, a_3, \dots$ be a sequence of positive integers which satisfies
\begin{equation*}
    \sum_{d|n} a_d = k^n 
\end{equation*}
for all $n \geq 1$. Prove that $n$ divides $a_n$ for all $n \geq 1$.

\solution{  Given a positive integer $k$, we note that the sequence $a_i$ is uniquely defined.

Here's a combinatorial solution:  Consider an alphabet with $k$ letters. Note that the number of strings of length $n$ is $k^n$. Now, define $f(d)$ as the number of strings of length $d$ with period $d$ (i.e. it cannot be expressed as a repeated string of smaller length, such as "\texttt{abcabc}"). By construction, every string of length $n$ can be paired up with the string representing its period (i.e. "\texttt{abcabc}" maps to "\texttt{abc}"). We therefore have
\begin{equation*}
    \sum_{d | n} f(d) = k^n
\end{equation*}
By uniqueness of $a_n$, this implies $a_n = f(n)$. Also, by construction of $f(n)$, we note that the strings of length $n$, period $n$ can be partitioned into sets of size $n$ by cycling each string through its $n$ distinct cyclic representations (i.e. "\texttt{abc}" yields $\{$ "\texttt{abc}", "\texttt{bca}", "\texttt{cab}" $\}$. Thus, we clearly note by construction that $n$ divides $f(n)$ and thus $n$ divides $a_n$. \qed

\textit{Remark:}  One can also use Mobius inversion to obtain
\begin{equation*}
    a_n = \sum_{d | n} \mu(d) k^d
\end{equation*}
Another approach is to use induction.

}


% Source: https://artofproblemsolving.com/community/c6t46036f6h1694575_prove_u_n_is_a_perfect_square
\item Let $a_0 = a_1 = 1$ and $a_{n+1} = 7a_n  - a_{n-1} - 2$ for all positive integers $n$. Prove that $a_n$ is a perfect square for all $n$.

\solution{

\includegraphics[width=0.7\textwidth]{numbertheory_sequence_solution.png}

}


% Source: https://artofproblemsolving.com/community/c6t46036f6h1377548_prove_all_primes_appear_in_the_sequence
% Original:  Thailand MO 2016,P5
\item Let $p_1 = 2$ and define a sequence of prime numbers $p_1, p_2, p_3, \dots$ such that, for all positive integers $n$, $p_{n+1}$ is the least prime factor of $n \cdot p_1^{1!} \cdot p_2^{2!} \dots p_n^{n!} + 1$. Prove that all primes appear in the sequence.

\solution{  Assume for contradiction the problem claim does not hold, and let $p$ be the smallest (odd) prime not appearing in the sequence. We thus have that $p$ is not the least prime factor of $1 + n \prod_{i=1}^n p_i^{i!}$ for all $i$. Let $C$ be the smallest integer such that all primes less than $p$ appear in $p_1, p_2, \dots, p_C$. Thus, if $p$ divides $1 + n \prod_{i=1}^n p_i^{i!}$ for some $i > C$, then it must be the smallest factor.

Therefore, $p$ does not divide $1 + n \prod_{i=1}^n p_i^{i!}$ for all $i > C$. Now, define $T_m = \prod_{i=1}^m p_i^{i!}$. By Fermat's Little, we have for sufficiently high $i$ and $p_i$, $p_i^{i!} \equiv 1$ (mod $p$). Thus, the residuce class of $T_m$ stays constant for sufficiently high $m$. Therfore, there exists an $m$ high enough such that $m T_m \equiv -1$ (mod $p$), which proves that $p$ divides $1 + m T_m$, which is a contradiction. \qed

}


\end{enumerate}


\section{Miscellaneous}
\begin{enumerate}

% Source: https://artofproblemsolving.com/community/c6t31442f6h1720333_two_variables_diophantine_with_natural_parameter
\item Let 
\begin{equation*}
    E(x,y)=\frac{x}{y} +\frac{x+1}{y+1} +\frac{x+2}{y+2} .
\end{equation*}
\begin{enumerate}
    \item Find all integers $x, y \in \mathbb{Z}$ such that $E(x, y) = 3$.
    \item Prove that there are infinitely many natural numbers $ n $ such that$ E(x,y)=n $ has at least one solution in $x, y \in \mathbb{Z}$
\end{enumerate}

\solution{
\begin{enumerate}
    \item If $x < y$, then clearly $E(x, y) < 3$. Likewise, if $x > y$, then clearly $E(x, y) > 3$. Thus $x = y$, which easily checks to be a solution.
    
    \item For any $k \in \mathbb{Z}$, let $n = 11k + 3$. Then $x = 6k+1$ and $y = 1$ yields a solution to $E(x, y) = n$.
\end{enumerate}
\qed
}


\item Prove that $m + n \leq \textrm{gcd}(m, n) + \textrm{lcm}(m, n)$ for all positive integers $m, n$. When does equality occur?

\solution{ Let $d = \textrm{gcd}(m, n)$, and denote $m = da$ and $n = db$. We note that $\textrm{lcm}(m, n) = mn/d$. Thus, dividing the given inequality by $d$ gives a sufficient inequality to prove is $a + b \leq 1 + ab$ for all positive itnegers $a, b$. However, this is simply $(a - 1)(b - 1) \leq 0$ which is clearly true for all $a, b \geq 1$. Equality occurs only if either $a = 1$ or $b = 1$, which occurs iff $m$ or $n$ equals $d$, which occurs iff $m$ divdes $n$ or $n$ divides $m$. \qed
}

% Source: https://artofproblemsolving.com/community/c6t177f6h1733020_numbers_that_divide_p1_for_primes_p
\item Let $a < b$ be natural numbers such that for all prime numbers $p > b$, at least one of $a$ and $b$ divides $p-1$. Prove that $a \leq 2$.

\solution{ Consider the arithmetic sequence $a_n = abn - 1$. By Dirichlet's theorem, this sequence contains infinitely many prime numbers (noting that clearly gcd($ab, 1) = 1$). Let $abk - 1$ be some prime number in this sequence. By the problem condition, either $a$ or $b$ divides $abk - 2$. However, as $a$ and $b$ both divide $abk$, this implies either $a$ or $b$ divides 2. Since $a < b$, this yields $a \leq 2$. \qed
}

% Source: https://artofproblemsolving.com/community/c6t177f6h1745667_easy_number_theoryi_think
\item Let $a = 222\dots2$ where the digit $2$ is denoted 2018 times. Prove that there are no positive integers $x, y$ such that $a = xy(x+y)$.

\solution{ Assume for contradiction such an $x, y \in \mathbb{Z}$ exists. We consider $a$ modulo 9: $a \equiv 2 \cdot 2018 = 4036 \equiv 4$ (mod 9). Thus 3 does not divide $a$, and therefore $3$ does not divide either $x$, $y$, or $x + y$. Thus, we may let $x = 3m \pm 1$ and $y = 3n \pm 1$ where the $\pm$ sign is the same for both $x$ and $y$. Note
\begin{align*}
    xy( x+y) &= ( 3m\pm 1)( 3n\pm 1)( 3m+3n\pm 2) \\
 &=( 9mn \pm 3( m+n) +1)( 3( m+n) \pm 2) \\
 &\equiv ( \pm 3( m+n) +1)( 3( m+n) \pm 2) \\
 &=\pm 9( m+n) +9( m+n) \pm 2 \\
 &\equiv \pm 2 \pmod 9
\end{align*}

As $\pm 2 \not \equiv 4$ (mod 9), this yields a contradiction. \qed

\textit{Remark:} The question can easily be generalised to any digit $d$ and length $l$ such that $dl \not \equiv \pm 2$ (mod 9)
}

% Source: 2017 IMOC N1
\item If $f : \mathbb{N} \to \mathbb{R}$ is a function such that 
\begin{equation*}
    \prod_{d | n} f(d) = 2^n
\end{equation*}
holds for all $n \in \mathbb{N}$, show that $f$ sends $\mathbb{N}$ to $\mathbb{N}$.

\solution{ First, assuming the existence of such a function, we note $f$ is unique. Letting $n = 1$ yields $f(1) = 2$. By induction, assuming the known values $f(1), f(2), \dots, f(k-1)$, letting $n = k$ yields a unique value for $f(k)$. \\

We now simply note that $f(n) = 2^{\phi(n)}$ is a valid solution which sends $\mathbb{N}$ to $\mathbb{N}$, noting the well-known result that $\sum_{d|n} \phi(d) = n$ \qed
}

\item Find all positive integers $n \geq 2$ such that $n^{n-1} - 1$ is square-free.

\solution{ First, note that if $n = 2$, then $n^{n-1} - 1 = 1$ which is not square-free. Now, consider $n$ odd. Thus $n$ is either 1 or 3 (mod 4). As $n-1$ is even, this implies $n^{n-1}$ is 1 (mod 4), and thus 4 divides $n^{n-1} - 1$, hence not square-free.

Now, assume $n \geq 4$ even, and consider some $p \mid (n-1)$. Note, $p$ is odd and clearly $p$ does not divide either $n$ or $1$. Thus, by lifting the exponent lemma, we have
\begin{equation*}
    v_p(n^{n-1} - 1) = v_p(n^{n-1} - 1^{n-1}) = v_p(n-1) + v_p(n-1) = 2 v_p(n-1) \geq 2
\end{equation*}
Hence $p^2$ divides $n^{n-1} - 1$ and is thus not square-free. Therefore, only $n=2$ satisfies the problem condition. \qed
}


%Source: https://artofproblemsolving.com/community/c6t177f6h573987_two_quadratics_have_integer_roots
% Original source: Indian RMO 2013 Paper 1 Problem 6

\item Let $m$ and $n$ be two integers such that both the quadratic equations $x^2 + mx - n = 0$ and $x^2 - mx + n = 0$ have integer roots. Prove that $n$ is divisible by $6$.

\solution{  Let $a, b$ be the two (not necessarily distinct) roots of  $x^2 + mx - n=0$ and let $c, d$ be the two (not necessarily distinct) roots of  $x^2 - mx + n=0$. By Vieta, we have $ab = -n$, $a+b = -m$ and $cd = n$, $c+d = m$. Now, adding the four equations
\begin{align*}
    a^2 + ma - n = 0, \quad b^2 + mb - n = 0, \\
    c^2 - mc + n = 0, \quad d^2 - md + n = 0,
\end{align*}
yields
\begin{align*}
    a^2 + b^2 + c^2 + d^2 + m(a+b-c-d) = 0 \implies a^2 + b^2 + c^2 + d^2 = 2m^2
\end{align*}
Assuming for contradiction that $n$ is odd, this implies all of $a, b, c, d$ odd, and thus $m$ is even. Therefore
\begin{equation*}
    a^2 + b^2 + c^2 + d^2 = 2m^2 \equiv 0 \quad (\textrm{mod } 8)
\end{equation*}
However, $a^2 \equiv b^2 \equiv c^2 \equiv d^2 \equiv 1$ (mod 8), which yields a contradiction. Thus $n$ is even. Now assume for contradiction $n$ is not divisible by 3. Thus, none of $a, b, c, d$ is divisible by 3, which implies $a^2 \equiv b^2 \equiv c^2 \equiv d^2 \equiv 1$ (mod 3). Thus
\begin{align*}
     2m^2 = a^2 + b^2 + c^2 + d^2 \equiv 4 \quad (\textrm{mod } 3) \implies m^2 \equiv 2 \quad (\textrm{mod } 3)
\end{align*}
which yields a contradiction. Thus, 3 divides $n$ and therefore $n$ is divisible by 6. \qed
}

% Source: https://artofproblemsolving.com/community/c6t328f6h79682_cant_be_power_of_2
% Original: APMO 1998
\item Show that for any positive integers $a$ and $b$, $(36a+b)(a+36b)$ cannot be a power of $2$.

\solution{ Assume for contradiction $(36a+b)(a+36b) = 2^l$ for some $l \leq 0$. Let $d = \textrm{gcd}(a, b)$ and let $a = dm$ and $b = dn$ with gcd($m, n) = 1$. Note that $d \mid 2^l$, thus $d = 2^k$ for some $k \geq 0$. Diving by $d^2$ on both sides yields $(36m + n)(36n + m) = 2^{l-2k}$. Note that $(36m+n)$ is a power of 2 not less than 36. As $4 \mid 36$, we have $4 \mid n$. By symmetry, this implies $4 \mid m$, which contradicts $m, n$ coprime. \qed

}


% Source: https://artofproblemsolving.com/community/c6t328f6h53874_sum_of_remainders
% Original: Iberoamerican Olympiad 2005
\item Denote by $a \bmod b$ the remainder of the euclidean division of $a$ by $b$. Determine all pairs of positive integers $(a,p)$ such that $p$ is prime and 
\begin{equation*}
    a \bmod p + a \bmod 2p + a \bmod 3p + a \bmod 4p = a + p.
\end{equation*} 

\solution{ Taking both sides modulo $p$, we obtain $3a \equiv 0$ (mod $p$), thus either $p = 3$ or $p \mid a$. We also note that $\textrm{LHS} \leq (p-1) + (2p-1) + 3p-1) + (4p-1)$. This bounds $a \leq 9p-4$.  Thus, if $p = 3$, then $a \leq 23$. Checking each case yields $a = 1$ or $a = 17$. Otherwise, as $a \leq 9p-4$ and $p | a$, this implies $a \in \{p, 2p, 3p, 4p, 5p, 6p, 7p, 8p\}$. Checking each case gives only $a = 3p$ as a solution. Thus the solutions are
\begin{equation*}
    (a, p) = (1, 3) \quad \textrm{ or } \quad (a, p) = (17, 3) \quad \textrm{ or } \quad a = 3p, p \textrm{ prime}
\end{equation*}
\qed


}


% Source: https://artofproblemsolving.com/community/c6t328f6h299_baltic99
% Original: Baltic Way 1999
\item Prove that there exist infinitely many even positive integers $k$ such that for every prime $p$ the number $p^2+k$ is composite.

\solution{ We note that any $k$ which is even, and $k \equiv 2$ (mod 3) and such that $k + 9$ is composite will suffice. Indeed, if $p = 2$, then $k + 4$ is even and $\geq 2$ and thus composite. If $p = 3$, then $k + 9$ is composite by definition. If $p > 3$, then $p^2 \equiv 1$ (mod 3). Thus $k + p^2 \equiv 0$ (mod 3), which implies $p^2 + k$ is composite. \\

Several standard arguments can be given to show infinitely many such $k$ exist. Indeed, the infinite arithmetic progression $k = 66l + 2$ works. One can also argue using the fact that there exist arbitrarily long consecutive sequences of composite numbers. \qed

}

% Source: https://artofproblemsolving.com/community/c6h531846p3039014
% and https://artofproblemsolving.com/community/c6t328f6h64244_perfect_square
% Original: Czech-Polish-Slovak 2002 Q4, Iran 2005 P1, MOSP 2004 Homework
\item An integer $n > 1$ and a prime $p$ are such that $n$ divides $p-1$, and $p$ divides $n^3 - 1$. Prove that $4p - 3$ is a perfect square.

\solution{ From $p \mid n^3 - 1$, we obtain $p \mid (n-1)(n^2 + n + 1)$. If $p \mid (n-1)$, then $p \leq n-1$, which contradicits $n \leq p - 1$. Thus $p \mid (n^2 + n + 1)$.  We let $p-1 = nm$. Thus $m \leq n+1$, since $nm + 1 \mid (n^2 + n + 1)$. \\

If $m<n+1$, take $m=n-k$. $p=n(n-k)+1 \implies p|n(k+1) \implies p|k+1 \implies p =nm+1 \le n-m+1 \implies nm \le n-m$ But this is a contradiction. \\

Thus $n = m+1$, which gives $p = n(n+1) + 1$ and thus $4p - 3 = 4(n^2 + n + 1) - 3) = (2n+1)^2$ which proves $4p-3$ is a perfect square. \qed

}


% Source: https://artofproblemsolving.com/community/c6t328f6h54902_find_primes
% Original: INMO 1995 Problem 6
\item Let $p$ be an odd prime. Find all primes $p$ for which the quotient 
\begin{equation*}
    \frac{2^{p-1} - 1 }{p}
\end{equation*}
is a square.

\solution{As $p$ is odd, let $p = 2k+1$, and assume that the given quotient is $a^2$. This yields $(2^{2k} - 1) = pa^2 \implies (2^k - 1)(2^k + 1) = pa^2$. Now, either $p$ divides $2^k - 1$ or $p$ divides $2^k + 1$. If $p$ divides $2^k - 1$, then
\begin{equation*}
    \left( \frac{2^k - 1}{p} \right) (2^k + 1) = x^2
\end{equation*}
with both factors on the left being integers. We also note that gcd($2^k - 1, 2^k + 1) = 1$, thus gcd($(2^k - 1)/p, 2^k + 1) = 1$. Therefore  $2^k + 1$ is a perfect square. Likewise, if $p$ divides $2^k + 1$, then $2^k - 1$ is a perfect square. \\

\textbf{Case 1}: $2^k + 1 = m^2$ for some $m \in \mathbb{Z}$. Then $2^k = (m-1)(m+1)$. We conclude $m = 3$, thus $p=7$.

\textbf{Case 2:} $2^k - 1 = m^2$ for some $m \in \mathbb{Z}$. Then $2(2^{k-1} - 1) = (m-1)(m+1)$. Note that $m$ is odd, thus $(m-1)(m+1) \equiv 0$ (mod 4). Assuming $k > 1$, we have $2(2^{k-1} - 1) \equiv 2$ (mod 4) which yields a contradiction. We thus only check $k = 1$, which gives $p = 3$.

Therefore, we have two solutions: $p = 3$ or $p = 7$.\qed

}

% Source: https://artofproblemsolving.com/community/c6t328f6h63029_prime_number
% Original: CWMO 2005-3  (uses the year 2005)
\item Set $S = \{1, 2, 3, ..., 2018\}$. If among any $n$ pairwise coprime numbers in $S$ there exists at least a prime number, find the minimum of $n$.


\solution{  Let us first consider $n = 15$. Note that the set $\{1, 2^2, 3^2, 5^2, 7^2, \dots, 43^2 \}$ consisting of 1 and the squares of the first 14 primes will work. Thus $n \geq 16$.  We now prove that amongst any 16 pairwise coprime elements, at least one is prime. 

Let $A = \{a_1, a_2, \dots, a_n\}$ be a set with $n \geq 16$ pairwise coprime elements, and order it in order of smallest prime dividing each element (if 1 is included, let $a_1 = 1$). Let $p_i$ be the smallest prime dividing $a_i$. As $A$ pairwise corpime, we have $p_2 < p_3 < p_4 < \dots p_n$. Thus, as $p_2 \geq 2$, we have $p_n \geq 47$ as 47 is the 15th smallest prime. Now assuming that $a_n$ is not prime, we have $a_n \geq p_n^2 = 2209 > 2018$ which yields a contradiction, thus $a_n$ is prime, which proves $n = 16$ is the minimal satisfying the problem condition. \qed


\textit{Remark:} Nothing special about 2018, can easily generalise. For an arbitrary $n$, just requires a manual computation to check the number of primes $p \leq \sqrt{n}$.
}

% Source: https://artofproblemsolving.com/community/c6t328f6h24081_prove_a_square
\item Let $m$ and $n$ be given positive integers such that $mn$ divides $m^2 + n^2 + m$. Prove that $m$ is a square of an integer.

\solution{ Assume for contradiction $m$ is not square. Thus, there exists a prime factor $p | m$ such that $p^{2k+1}$ divides $m$ and $p^{2k+2}$ does not divide $m$. Thus, $p^{2k+1}$ divides $m^2 + n^2 + m$, thus it divides $n^2$, thus $p^{k+1}$ divides $n$. Therefore $p^{2k+2}$ divides $mn$, and thus divides $m^2 + n^2 + mn$. However, $p^{2k+2}$ divides $m^2$ and $n^2$, and thus $p^{2k+2}$ divides $m$, contradiction. \qed

}

% Source: https://artofproblemsolving.com/community/c6t328f6h111591_totient_function_congruences_and_groups
\item Let $a, b$ be two positive integers such that gcd$(a, b) = 1$. Prove that
\begin{equation*}
    a^{\phi{(b)}} + b^{\phi{(a)}} \equiv 1 \quad (\textrm{mod } ab)
\end{equation*}

\solution{ By Euler-Phi, we have $a^{\phi{(b)}} \equiv 1$ (mod $b$) and $b^{\phi{(a)}} \equiv 1$ (mod $a$). Noting that $b | b^{\phi{(a)}}$ and $a | a^{\phi{(a)}}$, we obtain
\begin{equation*}
    a | a^{\phi{(b)}} + b^{\phi{(a)}} - 1 \quad \textrm{ and } \quad b | a^{\phi{(b)}} + b^{\phi{(a)}} - 1
\end{equation*}
As $a, b$ are coprime, this implies $ab | a^{\phi{(b)}} + b^{\phi{(a)}} - 1$, which thus proves the problem statement. \qed


}


% Source: https://artofproblemsolving.com/community/c6t328f6h60596_sum_of_prime_numbers
\item Determine all prime numbers $p$ such that $p$ is the sum of all primes less than $p$.

\solution{ The only solution is $p = 5$ (as $5 = 2 + 3$).  Let $S(n)$ denote the sum of all prime numbers less than $n$. Let $p_i$ denote the $i$-th prime number. Note that $S(p_3) = S(5) = 5$ and $S(p_4) = S(7) = 10 > 7 = p_4$. Now assume $S(p_i) > p_i$ for some $i$. Thus $S(p_{i+1}) = p_i + S(p_i) > 2p_i$. However, by Bertrand's postulate $2p_i \geq p_{i+1}$, thus $S(p_{i+1}) > p_{i+1}$. Therefore, by induction $S(p_i) \not = p_i$ for all $i \geq 4$. Thus, the only solution is $p = 5$. \qed
}

% Source: https://artofproblemsolving.com/community/c6t328f6h53472_2p3p
\item Let $p$ be a prime number. Prove that $2^p+3^p$ cannot be non-trivial perfect power (i.e. a positive integer of the form $a^b$ where $b > 1$).

\solution{ We simply apply lifting the exponent lemma. If $p = 2$, we have $2^2 + 3^2 = 13$ which is not a perfect power. Now assume $p$ is an odd prime. We have
\begin{equation*}
    v_5(2^p + 3^p) = v_5(2+3) + v_5(p) = 1 + v_5(p)
\end{equation*}
Now, if $p \not = 5$, then $v_5(2^p + 3^p) = 1$, which implies $2^p+3^p$ cannot be perfect power. Otherwise, if $p = 5$, then $2^p + 3^p = 275 = 5^2 \cdot 11$ which is not a perfect power. \qed

}

% Source: https://math.stackexchange.com/questions/2372582/this-is-a-canadian-mathematic-olympiad-question
% Original: Canadian Maths Olympiad
% This was labelled under number theory, although its more like combinatorics
\item Let $n$ be a positive integer and define $S_n = \{1, 2, 3, \dots, n\}$. We denote a non-empty subset $T$ of $S_n$ as \textit{balanced} if the median of $T$ is equal to the average of $T$. For each $n \geq 1$, prove that the number of balanced subsets of $S_n$ is odd.

\solution{
Define a subset $U$ of $S_n$ as \textit{unbalanced} if it is not balanced. We define a map $f : P(S_n) \to P(S_n)$ given by $U$ maps to  $\{ n -k+1 \;:\; k \in U  \}$ (i.e. it's a reversal map sending each element $k$ in $U$ to $n-k+1$). Note we clearly have $f(f(U)) = U$ for all $U \in P(S_n)$. Furthermore, the map flips the relative order of the mean and median of $U$. Thus, if $U$ is unbalanced, then $f(U) \not = U$. \\

Therefore, we have paired up each unbalanced set uniquely, proving that there are an \textit{even} number of unbalanced sets. As the total number of non-empty subsets is $2^n-1$, there are thus an \textit{odd} number of balanced subsets. \qed

}

% Source: https://math.stackexchange.com/questions/1682542/2009-benelux-math-olympiad-bxmo-number-theory-problem
% Original: 2009 Benelux Math Olympiad (BxMO)
\item Let $n$ be a positive integer and let $k$ be an odd positive integer. Moreover, let $a$, $b$ and $c$ be integers (not necessarily positive) satisfying the equation
\begin{equation*}
    a^n + kb = b^n + kc = c^n + ka
\end{equation*}

Prove that $a=b=c$.

\solution{ Note that, if any two of $a, b, c$ are equal, then so is the third. Let us now assume for contradiction that $a, b, c$ are all distinct. We thus obtain the equations:
\begin{equation*}
    k = \frac{b^n - a^n}{b - c} = \frac{c^n - b^n}{c-a} = \frac{a^n - c^n}{a - b}
\end{equation*}
By Pigeonhole principle, at least two of $a, b, c$ must have the same parity. Wlog, assume $a \equiv b$ (mod 2). Then, from $k = (b^n - a^n)/(b-c)$, noting that $k$ is odd, we have $b \equiv c$ (mod 2) and thus require that $a$ and $c$ have the same parity. Thus $a \equiv b \equiv c$ (mod 2).

Now, by the same argument, at least two of $a, b, c$ leave the same remainder mod 4. Wlog, assume $a \equiv b$ (mod 4). Again, from $k = (b^n - a^n)/(b-c)$, we require that $b-c$ divisible by 4, and thus $b$ and $c$ leave the same remainder mod 4 since $k$ must be an odd integer. \\

We therefore obtain by induction that $a \equiv b \equiv c$ (mod $2^k$) for all $k \geq 1$. Thus, this must imply $a = b = c$. \qed


}


\end{enumerate}



\end{document}
