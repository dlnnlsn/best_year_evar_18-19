\documentclass{article}

\usepackage{amsmath,amsthm,amssymb}
% \usepackage{fullpage}
\usepackage{enumerate}
\usepackage{hyperref}

\DeclareMathOperator{\lcm}{lcm}


\begin{document}

\begin{center}
\textbf{\Large Senior January Monthly Problem Set}
\\ \vspace{1em}
\textbf{\large Due: 18 January 2019}
\end{center}

\vspace{24pt}

\begin{enumerate}[1.]

\item %SH-2006-1
$P$, $Q$ and $R$ are any points on $BC$, $CA$ and $AB$ respectively of a triangle $ABC$. Let the centres of the circumcircles $AQR$, $BRP$ and $CPQ$ be $X$, $Y$ and $Z$. Prove that triangles $XYZ$ and $ABC$ are similar.


\vspace{6pt}
\item % SW-2007-3
Steve determines the geometric mean of two positive integers in the following way:
\begin{enumerate}
	\item He writes them down in their decimal representation, one below the other, and prepends zeros to the smaller number (if applicable) such that their lengths are equal.
	\item He determines the geometric mean of each pair of digits below each other. If the result is not an integer, only the integer part is used.
	\item The digits determined by this procedure form the result.
\end{enumerate}
Determine all pairs $(a,b)$ of positive integers for which Steve's procedure yields the correct result. (For example, one such pair is $(12; 48)$.)


\vspace{6pt}
\item % JM-2009-3
\begin{enumerate}
	\item Prove that if $p > 10$ is a prime number that divides $a^4+a^3+a^2+a+1$ for some integers $a$, then $p$'s decimal expansion ends in a $1$.
	\item For any prime $p$ whose decimal expansion ends in a $1$, and any positive integer $k$, prove that there exists an integer $a$ such that $p^k$ divides $a^4+a^3+a^2+a+1$.
\end{enumerate}


\vspace{6pt}
\item % SW-2008-1
The set $S$ of nonnegative integers has the property that every nonnegative integer $n$ can be uniquely written as $n = a+2b$ where $a,b \in S$ are not necessarily distinct. How many elements of $S$ are less than $2018$?



\vspace{6pt}
\item % DB-2012-3
Jacob has a balance scale and wishes to buy weights from Sipho. Sipho tells Jacob that he sells weights in the following way: Jacob has to specify a sequence of $n$ integers $a_1, a_2, \dotsc a_n$, and then Sipho will make 1 weight of mass $a_1$, two weights of mass $a_2$, etc., and $n$ weights of mass $a_n$.

What is the largest $k$ for which Jacob can specify some sequence $(a_1, \dotsc, a_n)$ and still be able to measure every integral weight from $1$ to $k$? (For example, with weights with mass $4$ and $7$, he can measure a weight of $3$ by putting one weight on the one side and the other on the other side of the balance scale.)


\vspace{6pt}
\item
Does there exist a natural number $n$ such that
\[
	1^{2018} + 2^{2018} + \cdots + n^{2018}
\]
is prime?


\vspace{6pt}
\item % Modified version of IMO1999 Problem 2
Fix a natural number $n \geq 2$. Find the smallest constant $C$ such that
\[
	 \sum_{1 \leq i < j \leq n} x_i x_j (3x_i^2 + x_j^2)(x_i^2 + 3x_j^2) \leq C {\left( \sum_{i = 1}^{n} x_i \right)}^6
\]
for all non-negative real numbers $x_1, x_2, \dots, x_n$. For this value of $C$, when does equality occur?


\item Let $ABC$ be a triangle with circumcircle $\Omega$. Let $P$, $Q$ be 2 points not on $\Omega$ such that the line $PQ$ passes through the centre of $\Omega$. Let $D$, $E$, $F$ be the feet of the perpendiculars from $P$ to $BC$, $CA$, $AB$, and let $X$, $Y$, $Z$ be the feet of the perpendiculars from $Q$ onto the same sides, respectively. Prove that the perpendiculars from $D$, $E$, $F$ to $YZ$, $ZX$, $XY$ are concurrent.

\end{enumerate}


\vfill
\textbf{\Large Email submission guidelines}
\begin{itemize}
	\item Email your solutions to \href{mailto:samf.training.assignments@gmail.com}{\texttt{samf.training.assignments@gmail.com}}.
	\item Submit each question in a single separate PDF file (with multiple pages if necessary), with your name and the question number written on each page.
	\item If you take photographs of your work, use a document scanner such as CamScanner to convert to PDF.
	\item If you have multiple PDF files for a question, combine them using software such as PDFsam.
\end{itemize}

\end{document}
