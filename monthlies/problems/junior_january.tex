\documentclass{article}

\usepackage{amsmath,amsthm,amssymb}
\usepackage{fullpage}
\usepackage{enumerate}

\DeclareMathOperator{\lcm}{lcm}


\begin{document}

\begin{center}
\textbf{\Large Junior January Monthly Problem Set}
\\ \vspace{1em}
\textbf{\large Due: 18 January 2019}
\end{center}

\begin{enumerate}[1.]

\vspace{6pt}
\item % IPMO 2006 Junior Team Q4
\[ \sqrt[3]{\frac{2+1}{2}\cdot\frac{3+1}{3}\cdot\frac{4+1}{4}\dotsm\frac{a+1}{a}} = 4. \]
Find the value of $a$.


\vspace{6pt}
\item % IPMO 2005 Junior Team Q3
A circle of radius $1$ is centred at the origin. Two particles start moving at the same time from the point $(1,0)$ and move around the circle in opposite directions. One of the particles moves counterclockwise with constant speed $\nu$ and the other moves clockwise with constant speed $3\nu$. After leaving $(1,0)$, the two particles meet first at point $P$, and continue until they meet next at point $Q$. Determine the coordinates of the point $Q$.


\vspace{6pt}
\item % IPMO 2005 Junior Team Q6
Determine the number of triplets $(k,l,m)$ of positive integers such that
\begin{align*}
  k+l+m &= 97 \quad \mathrm{and} \\
  \frac{4k}{5} +\frac{5l}{6} +\frac{6m}{7} &= 82.
\end{align*}


\vspace{6pt}
\item 


\vspace{6pt}
\item 


\vspace{6pt}
\item


\vspace{6pt}
\item


\vspace{6pt}
\item % 


\end{enumerate}


\vfill
\textbf{\Large Email submission guidelines}
\begin{itemize}
	\item Email your solutions to \verb!samf.training.assignments@gmail.com!.
	\item Submit each question in a single separate PDF file (with multiple pages if necessary), with your name and the question number written on each page.
	\item If you take photographs of your work, use a document scanner such as CamScanner to convert to PDF.
	\item If you have multiple PDF files for a question, combine them using software such as PDFsam.
\end{itemize}

\end{document}
