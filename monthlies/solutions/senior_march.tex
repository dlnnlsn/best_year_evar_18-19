\documentclass{article}

\usepackage{amsmath,amssymb}
\usepackage{fullpage}
\usepackage{enumerate}
\usepackage{hyperref}
\usepackage{tikz,tkz-euclide}
\usepackage{float}

\begin{document}

\begin{center}
\textbf{\Large Senior March Monthly Problem Set}
\\ \vspace{1em}
\textbf{\large Solutions}
\end{center}

\begin{enumerate}[1.]

\item % SW-2010-4
{\itshape Prove that the inequality
\[ \frac{x}{y} +\frac{y}{z} +\frac{z}{x} -\frac{x}{z} -\frac{z}{y} -\frac{y}{x} < \frac{1}{4xyz} \]
holds for all real numbers $x,y,z \in (0,1)$.}

Note that
\begin{align*}
    \frac{x}{y} +\frac{y}{z} +\frac{z}{x} -\frac{x}{z} -\frac{z}{y} -\frac{y}{x} &< \frac{1}{4xyz} \\
    \iff \quad x^2z + y^2x + z^2y - x^2y - z^2x - y^2 z &< \frac{1}{4} \\
    \iff \quad (x-y)(x-z)(z-y) &< \frac{1}{4}
\end{align*}

Note, if $x=y$ or $y=z$ or $z=x$, then the inequality trivially holds.  Furthermore, note that the inequality is cyclic.  Thus, we may assume wlog that $x > y$ and $x > z$.

If $y > z$, then we have $(x-y)(x-z)(z-y) < 0 < \frac{1}{4}$, which trivially proves the inequality.  We therefore consider the case where $z > y$.  By AM-GM, we have
\begin{equation*}
    \sqrt{(x-z)(z-y)} \leq \frac{(x-z) + (z-y)}{2} = \frac{x-y}{2} < \frac{1}{2}
\end{equation*}
and thus, by squaring, we obtain
\begin{equation*}
    (x-z)(z-y) < \frac{1}{4}
\end{equation*}
Finally, noting that $x - y < 1$, we conclude that
\begin{equation*}
    (x-y)(x-z)(z-y) < (x-z)(z-y) < \frac{1}{4}
\end{equation*}
which proves the inequality.

\vspace{6pt}
\item % DB-2012-5
{\itshape In the game Memory you are given $2n$ cards, where $n$ is a given positive integer. The cards start lying face down in an array on the table. On the face of each card there is a picture. There are $n$ different pictures, each occurring on exactly two of the cards. In a turn you may choose two cards and then turn them both face-up. If they have the same picture, you may remove them from the table. Otherwise you turn them face-down again. Your goal is to clear all the cards from the table.

What is the least integer $k$ for which it is always possible to finish the game in at most $k$ turns?}


\vspace{6pt}
\item % DB-2012-8
{\itshape Prove that there are infinitely many integers n such that both the arithmetic mean of its divisors and the geometric mean of its divisors are integers.

(Recall that for $k$ positive real numbers $a_1, a_2, \dotsc, a_k$, the arithmetic mean is $\frac{a_1 +a_2 +\dotsb +a_k}{k}$ and the geometric mean is $\sqrt[k]{a_1 a_2 \dotsm a_k}$.)
}

Let $p$ be a prime such that $p \equiv 1$ (mod 3), and let $n = p^2$.  Note that there are three divisors of $n$, which are $1, p, p^2$.

The geometric mean is therefore $\sqrt[3]{1 \cdot p \cdot p^2} = \sqrt[3]{p^3} = p$ which is clearly an integer.  The arithmetic mean is $\frac{1 + p + p^2}{3}$, however, note that $1 + p + p^2 \equiv 1 + 1 + 1 \equiv 0$ (mod 3), thus $\frac{1 + p + p^2}{3}$ is also an integer.

Thus, $n$ satisfies the problem conditions.  Finally, note that by Dirichlet's theorem, since gcd(1, 3) = 1, there are infinitely many primes $p$ such that $p \equiv 1$ (mod 3), and thus infinitely many $n$ satisfying the problem condition. 


\vspace{6pt}
\item % The Andrew
{\itshape Let $\mathbb{P}$ be the set of points in the Euclidean plane, and $O \in \mathbb{P}$ be a given point. Let $\mathbb{P}_O = \mathbb{P} \backslash \{O\}$ be the set of points in the Euclidean plane excluding $O$.

Find all functions $f : \mathbb{P}_O \to \mathbb{P}_O$ satisfying both of the following conditions:
\begin{itemize}
  \item If $C \subset \mathbb{P}_O$ is a circle, then $f(C) = \{ f(P) \mid P \in C \}$ is also a circle.
  \item For any point $P \in \mathbb{P}_O$ we have that $O$, $P$ and $f(P)$ are collinear.
\end{itemize}}


\vspace{6pt}
\item % The Robin 
{\itshape Find all functions $f : \mathbb{R} \to \mathbb{R}$ such that there exists a strictly monotone function $g : \mathbb{R} \to \mathbb{R}$ which satisfies
\[ f(x)g(y) + g(x) = g(x+y) \]
for all $x, y \in \mathbb{R}$.}

We can equate the two following identities
$$ f(x)g(y) + g(x) = g(x+y) \quad \textrm{and} \quad f(y)g(x) + g(y) = g(y+x) $$
to obtain
\begin{equation} \label{eq:q5p1}
    g(x) (f(y) - 1) = g(y) (f(x) - 1) 
\end{equation}
Now, letting $y = 0$ in the original condition yields
\begin{align*}
    f(x) g(0) + g(x) = g(x) \implies f(x) g(0) = 0
\end{align*}
which implies either $f$ identically 0, or $g(0) = 0$.  However, $f$ identically 0 would imply that $g(x+y) = g(x)$ which implies $g$ constant, thus contradicting the fact that $g$ is strictly monotone.

Therefore $g(0) = 0$, and since $g$ strictly monotone, this implies $g(x) \not = 0$ for all $x \not = 0$.  Therefore, from (\ref{eq:q5p1}), we obtain
\begin{equation*}
    \frac{f(x) - 1}{g(x)} = \frac{f(y) - 1}{g(y)} \quad \textrm{for all } x, y \not = 0
\end{equation*}
Therefore, there exists a $c \in \mathbb{R}$ such that $f(x) - 1 = cg(x)$ for all $x \not = 0$.  Furthermore, substituting $x = 0$, $y = 1$ into (\ref{eq:q5p1}), we obtain $0 = g(1)(f(0) - 1)$, and since $g(1) \not = 0$, this implies $f(0) = 1$.

Therefore, we have that $f(x) - 1 = cg(x)$ for all $x \in \mathbb{R}$. Substituting this into the original equation, we obtain
\begin{align*}
    f(x)g(y) + g(x) &= g(x+y) \\
    \implies \quad (1 + cg(x))g(y) + g(x) &= g(x+y) \\
    \implies \quad 1 + c(1 + cg(x))g(y) + cg(x) &= 1 + cg(x+y) \\
    \implies \quad (1 + cg(x))(1 + cg(y)) &= 1 + cg(x+y) \\
    \implies \quad f(x)f(y) &= f(x+y)
\end{align*}

First, let's consider the case $c = 0$. This implies $f(x) = 1$, which fulfils the problem condition where $g(x) = x$.

Now, assume $c \not = 0$. As $g$ strictly monotonic, this implies $f$ strictly monotonic.  Furthermore, noting that $f(x) = f\left(\frac{x}{2} + \frac{x}{2} \right) = f^2\left(\frac{x}{2} \right)$, we have $f(x) \geq 0$ for all $x \in \mathbb{R}$.  But since $f$ strictly monotonic, this implies $f(x) > 0$ for all $x \in \mathbb{R}$.

We may therefore define the function $h : \mathbb{R} \to \mathbb{R}$ where $h(x) = \ln{f(x)}$ for all $x \in \mathbb{R}$. Note that
$$
h(x) + h(y) = \ln{f(x)} + \ln{f(y)} = \ln{f(x) \cdot f(y)} = \ln{f(x+y)} = h(x+y)
$$
Therefore, $h$ satisfies the Cauchy condition $h(x+y) = h(x) + h(y)$. Since $h$ is monotonic, this implies that $h$ must be linear. Thus, there exists $k \in \mathbb{R}$ such that $h(x) = kx$ for all $x \in \mathbb{R}$.  Thus $f(x) = e^{kx}$ and $g(x) = \frac{e^{kx} - 1}{c}$.

One can easily check that this satisfies the given condition.  In conclusion, the set of valid functions for $f$ are $f(x) = a^x$ where $a > 0$.




\vspace{6pt}
\item % 2017 Kurschak Competition Q3
{\itshape We put a number in each field of an $n \times n$ table $T$ such that no number appears twice in the same row. Prove that it is possible to rearrange the numbers in $T$ in such a way that each row of the rearranged table $T^*$ contains the same numbers that the corresponding row of $T$ contained, and moreover no number appears twice in the same column of $T^*$.
}

\vspace{6pt}
\item % The Robin
{\itshape Find all positive integers $m, n$ such that
\[ m^{2019} - m! = n^{2019} - n! . \]}

Firstly, we note that $m = n$ clearly yields a solution.  Now, let $f : \mathbb{N} \to \mathbb{Z}$ be a function such that $f(n) = n^{2019} - n!$ for all $n \in \mathbb{N}$. We shall prove that $f$ is injective, which proves that there exist no further solutions.

Assume for contradiction that there exists $m, n \in \mathbb{N}$ such that $f(m) = f(n)$ and $m > n$.  Note that clearly $n \not = 1$, thus we may assume $n > 1$.  Let $p$ be a prime divisor of $n$. Since $m > n$, we have $p$ divides $m!$.  Thus, since $m! - n! = m^{2019} - n^{2019}$, we have $p$ divides $m$


\vspace{6pt}
\item % Ukraine NMO 2018 Q11.8
{\itshape Let $ABC$ be an acute angled triangle with incentre $I$. Let $AI$ and $CI$ have midpoints $M$ and $N$ respectively and intersect $BC$ and $BA$ at $A^\prime$ and $C^\prime$ respectively. Let $K$ and $L$ be points inside triangles $AC^\prime I$ and $A^\prime CI$ respectively such that $\angle AKI = \angle AIC = \angle CLI$, $\angle AKM = \angle ICA$ and $\angle IAC = \angle CLN$. Show that the radii of the circumcircles of $LIK$ and $ABC$ are equal.
}

  
\end{enumerate}

\end{document}
