\documentclass{article}

\usepackage{amsmath,amsthm,amssymb}
\usepackage{fullpage}
\usepackage{enumerate}
\usepackage{graphicx}

\DeclareMathOperator{\lcm}{lcm}


\begin{document}

\begin{center}
\textbf{\Large Senior January Monthly Problem Set}
\\ \vspace{1em}
\textbf{\large Due: 18 January 2019}
\end{center}

\begin{enumerate}[1.]

\item %SH-2006-1
\textit{$P$, $Q$ and $R$ are any points on $BC$, $CA$ and $AB$ respectively of a triangle $ABC$. Let the centres of the circumcircles $AQR$, $BRP$ and $CPQ$ be $X$, $Y$ and $Z$. Prove that triangles $XYZ$ and $PQR$ are similar.}

Let the circumcircle of $ARQ$ and the circumcircle of $BPR$ intersect at $O$. Then we have that $\angle QOR = 180^\circ - \angle A$ and $\angle ROP = 180^\circ - \angle B$. It follows that $\angle POQ = 360^\circ - (180^\circ - \angle A) - (180^\circ - \angle B) = \angle A + \angle B = 180^\circ - \angle C$, and so $O$ lies on the circumcircle of $CQP$.

Let $D$ be the midpoint of $RO$ and let $E$ be the midpoint of $OQ$. Since $X$ and $Y$ are the circumcenters of $ARQ$ and $BPR$, and $OR$ is a common chord in these two circles, we know that $XY$ passes through $D$, and that $XY \perp RO$. Similarly, $XZ$ is a perpendicular bisector of $OQ$. We this have that $\angle X = 180^\circ - \angle EOD = \angle A$. A similar argument shows that $\angle Y = \angle B$, and hence $ABC$ is similar to $XYZ$.

\begin{figure}[!hb]
\centering
\includegraphics[width=0.6\linewidth]{senior_january_q1.mps}
\caption{Diagram for Problem 1}
\end{figure}


\vspace{6pt}
\item \label{problem:geometric_mean} % SW-2007-3
\textit{Steve determines the geometric mean of two positive integers in the following way:
\begin{enumerate}
	\item He writes them down in their decimal representation, one below the other, and prepends zeros to the smaller number (if applicable) such that their lengths are equal.
	\item He determines the geometric mean of each pair of digits below each other. If the result is not an integer, only the integer part is used.
	\item The digits determined by this procedure form the result.
\end{enumerate}
Determine all pairs $(a,b)$ of positive integers for which Steve's procedure yields the correct result. (For example, one such pair is $(12; 48)$.)}

Let
\[
  a = \sum_{k = 0}^{n} a_k 10^k \quad\text{and}\quad b = \sum_{k = 0}^{n} b_k 10^k
\]
be the decimal expansions of $a$ and $b$, possibly with leading zeros. Then Steve's procedure for calculating the geometric mean gives us
\[
  \sum_{k = 0}^{n} \left\lfloor \sqrt{a_k b_k} \right\rfloor 10^k.
\]
We thus require that
\[
  \left(\sum_{k = 0}^{n} \left\lfloor \sqrt{a_k b_k} \right\rfloor 10^k\right)^2 = ab = \left( \sum_{k = 0}^{n} a_k 10^k \right) \left( \sum_{k = 0}^{n} b_k 10^k \right).
\]

But we have that
\[
  \left(\sum_{k = 0}^{n} \left\lfloor \sqrt{a_k b_k} \right\rfloor 10^k\right)^2 \leq \left( \sum_{k = 0}^{n} \sqrt{a_k b_k} 10^k \right)^2 \leq  \left( \sum_{k = 0}^{n} a_k 10^k \right) \left( \sum_{k = 0}^{n} b_k 10^k \right)
\]
by the Cauchy-Schwarz inequality. Thus equality occurs if and only if 
\[
	\lfloor \sqrt{a_k b_k} \rfloor = \sqrt{a_k b_k}
\] 
for each $k$, and equality occurs in the application of the Cauchy-Schwarz inequality.

We thus require that $a_k b_k$ is a square for each $k$, and that either $a = 0$, or there is a real number $\lambda$ such that $b_k = \lambda a_k$ for each $k$. If $a \neq 0$ then we have that $a_k b_k = \lambda a_k^2$ for each $k$, and so $\lambda$ is the square of a rational number. Noting that $b_k$ is an integer for each $k$ and that the ration $b_k / a_k$ is the square of a rational number for each $k$, Table \ref{tbl:digits} shows the allowed values of $\lambda$ for each digit that may occur in $a$.

\begin{table}
\centering
\caption{Allowed ratio between the digits of $b$ and the digits of $a$. (Problem \ref{problem:geometric_mean})}
\label{tbl:digits}
\begin{tabular}{|c|c|c|c|c|c|}
\hline
  Digit & Ratio & Digit & Ratio & Digit & Ratio \\
\hline
0 & Any & 1 & 0, 1, 4, 9 & 2 & 0, 1, 4 \\
3 & 0, 1 & 4 & 0, 1/4, 1, 9/4 & 5 & 0, 1 \\
6 & 0, 1 & 7 & 0, 1 & 8 & 0, 1/4, 1 \\
9 & 0, 1/9, 4/9, 1 & & & & \\
\hline
\end{tabular}
\end{table}

Combining the above, we see that all solutions are given by the pairs $(a, b)$ such that one of the following conditions hold:
\begin{itemize}
\item $a$ is any natural number and $b = a$.
\item $a$ contains only the digits $0$, $1$ and $2$ and $b = 4a$.
\item $b$ contains only the digits $0$, $1$ and $2$ and $a = 4b$.
\item $a$ contains only the digits $0$ and $1$ and $b = 9a$.
\item $b$ contains only the digits $0$ and $1$ and $a = 9b$.
\item $a$ contains only the digits $0$ and $4$ and $b = 9a/4$.
\item $b$ contains only the digits $0$ and $4$ and $a = 9b/4$.
\end{itemize}


\vspace{6pt}
\item % JM-2009-3
\textit{\begin{enumerate}
	\item Prove that if $p > 10$ is a prime number that divides $a^4+a^3+a^2+a+1$ for some integers $a$, then $p$'s decimal expansion ends in a $1$.
	\item For any prime $p$ whose decimal expansion ends in a $1$, and any positive integer $k$, prove that there exists an integer $a$ such that $p^k$ divides $a^4+a^3+a^2+a+1$.
\end{enumerate}}

\begin{enumerate}

\item Note that $p$ divides $(a - 1)(a^4 + a^3 + \cdots + 1) = a^5 - 1$. Thus the order of $a$ modulo $p$ divides $5$, and hence is equal to $1$ or $5$. If the order of $a$ modulo $p$ is $1$, then $a \equiv 1 \pmod p$, giving us that $0 \equiv a^4 + a^3 + \cdots + 1 \equiv 1 + 1 + \cdots + 1 \equiv 5 \pmod p$, and so $p = 5$, a contradiction. Thus the order of $a$ modulo $p$ is $5$. Since $a^{p - 1} \equiv 1 \pmod p$ by Fermat's little theorem, this implies that $5 \mid p - 1$. We thus have that $p \equiv 1 \pmod 5$ and $p \equiv 1 \pmod 2$ (since $p > 10$), and so $p \equiv 1 \pmod{10}$.

\item Suppose that $p \equiv 1 \pmod 5$. Then there is a natural number $b$ such that the order of $b$ modulo $p$ is $5$. (For example, if $g$ is a primitive root modulo $p$, then $g^{(p - 1)/5}$ is such a number.)

We thus have that $p \mid b^5 - 1$, but $p \nmid b - 1$. Let
\[
	a = b^{p^{k - 1}}.
\]

By the Lifting the Exponent Lemma, we have that $p^k$ divides $a^5 - 1$. However, $p$ does not divide $a - 1$, and so $p^k$ divides
\[
	\frac{a^5 - 1}{a - 1} = a^4 + a^3 + \cdots + 1.
\]

\end{enumerate}


\vspace{6pt}
\item % SW-2008-1
\textit{The set $S$ of nonnegative integers has the property that every nonnegative integer $n$ can be uniquely written as $n = a+2b$ where $a,b \in S$ are not necessarily distinct. How many elements of $S$ are less than $2018$?
}

Let $T$ be the set of non-negative integers $n$ such that when $n$ is written in base $4$, every digit is equal to $0$ or $1$. We first note that $T$ satisfies the conditions in the problem. To see this, consider an arbitrary natural number $n$. Let
\[
	n = \sum_{k = 0}^{\infty} d_k 4^k
\]
be the base $4$ expansion of $n$. For each $k$, if $d_k = 0$, then let $a_k = b_k = 0$. If $d_k = 1$, let $a_k = 1$ and $b_k = 0$. If $d_k = 2$, let $a_k = 0$ and $b_k = 1$, and if $d_k = 3$, let $a_k = b_k = 1$. We see that the natural numbers
\[
	a = \sum_{k = 0}^{\infty} a_k 4^k \quad \text{and} \quad b = \sum_{k = 0}^{\infty} b_k 4^k
\]
are in $T$ and satisfy $a + 2b = n$. We now show that this representation is unique. Suppose that $a, b \in T$ are such that $a + 2b = n$, and let
\[
	a = \sum_{k = 0}^{\infty} a_k 4^k \quad \text{and} \quad b = \sum_{k = 0}^{\infty} b_k 4^k
\]
be the base $4$ expansions of $a$ and $b$. We note that there are no carries when calculating $a + 2b$ in base $4$, and so for each $k$ we must have that $a_k + 2b_k = d_k$. It is easy to check that for each possible value of $d_k$, there is a unique value for $a_k$ and $b_k$ such that this relation holds.

We will now show that $S = T$. We prove by strong induction that for each natural number $n$, that $S \cap \{0, 1, 2, \dots, n\} = T \cap \{0, 1, 2, \dots, n\}$. We note that since we must be able to write $0 = a + 2b$ for some $a, b \in S$, that we must have that $0 \in S$. Thus the claim holds for $n = 0$. Suppose that the claim holds for all natural numbers smaller than $n$. We show that $n$ is in $S$ if and only if $n$ is in $T$.

Suppose that $n \in S$. If $n \not\in T$, then we know that we can write $n = a + 2b$ for some $a, b \in T$. Since $n \not\in T$, we must have that $a, b < n$, and so $a, b \in S$ by the induction hypothesis. But then then we have two ways of writing $n$ as $a + 2b$ with $a, b \in S$: we can write $n = n + 2 \cdot 0$, and $n = a + 2b$, contradicting the uniqueness of this representation. Thus $n \in T$.

Conversely, suppose that $n \not\in S$, but $n \in T$. Since $n \not\in S$, we must be able to write $n = a + 2b$ with $a, b \in S$ and $a, b < n$. (Since we can not represent $n$ as $n + 2 \cdot 0$.) Since $a, b < n$, this implies that $a, b \in T$, and so this representation holds in $T$ as well. But we then have two ways of expressing $n$ as $a + 2b$ where $a, b \in T$: as $n = n + 2 \cdot 0$, and as $n = a + 2b$. We thus have that $n \not\in T$, and so the claim follows by induction.

The elements of $S$ are thus precisely those where each digit in base $4$ is either $0$ or $1$. Since $2018$ in base $4$ is ${133202}_{4}$, we deduce that the number of elements in $S$ smaller than $2018$ is $2^6 = 64$.



\vspace{6pt}
\item % DB-2012-3
\textit{Jacob has a balance scale and wishes to buy weights from Sipho. Sipho tells Jacob that he sells weights in the following way: Jacob has to specify a sequence of $n$ integers $a_1, a_2, \dotsc a_n$, and then Sipho will make 1 weight of mass $a_1$, two weights of mass $a_2$, etc., and $n$ weights of mass $a_n$.}

\textit{What is the largest $k$ for which Jacob can specify some sequence $(a_1, \dotsc, a_n)$ and still be able to measure every integral weight from $1$ to $k$? (For example, with weights with mass $4$ and $7$, he can measure a weight of $3$ by putting one weight on the one side and the other on the other side of the balance scale.)}


We claim for any $n$, the maximum $k$ such that one can specify some sequence $(a_1, a_2, \dots, a_n)$ and be able to measure every integral weight from $1$ to $k$ is
\[
  k_n = \frac{(2n + 1)!! - 1}{2},
\]
where $m!!$ denotes the product of the odd integers from $1$ to $m$.

We first show that $k_n$ is an upper bound for $k$. For some $i$, suppose that we put $s$ of the weights with mass $a_i$ on the left hand side of the scale, and $t$ of the weights with mass $a_i$ on the right hand side of the scale. If $s > t$, then this is equivalent to placing $s - t$ weights on the left hand side of the scale, and if $t > s$, then this is equivalent to placing $t - s$ of the weights on the right hand side of the scale. We can thus assume that for each $i$, all of the weights with mass $a_i$ are placed on the same side of the scale.

We now count how many distinct weights we can measure. We note that there are $2i + 1$ options for how we place the weights with mass $a_i$ on the scale: We can put any number from $1$ to $i$ on the left hand side of the scale, or we can put any number from $1$ to $i$ on the right hand side of the scale, or we can place no weights with mass $a_i$ on the scale.

This implies that there are
\[
  \prod_{i = 1}^{n} (2i + 1) = (2n + 1)!!
\]
possible ways of placing the weights on the scale. However, we have counted each weight that we can measure twice. (Except for $0$, which we measure by placing no weights on the scale.) This is because we can swap which side of the scale each of the weights which are on the scale is on. We thus see that we can measure at most
\[
  \frac{(2n + 1)!! - 1}{2}
\]
distinct non-zero weights. This number is therefore an upper bound for $k$, since if we can measure all of the integral weights from $1$ to $k$, then we can measure at least $k$ distinct non-zero weights.

We now show that it is possible to measure each weight from $1$ to $k_n$ with a suitable choice of the specified sequence $(a_1, a_2, \dots, a_n)$.

Let Jacob specify the sequence $(a_1, a_2, \dots, a_n)$ where $a_i = (2i - 1)!!$. We proceed by induction on $n$ to show that he can then measure every integral weight from $1$ to $k_n$. This is clearly true for $n = 1$. Suppose that it is true for some $n$. We show that it is true for $n + 1$ as well.

We first show that
\[
  k_n = 1 \cdot 1!! + 2 \cdot 3!! + \cdots + n \cdot (2n - 1)!!
\]
for all $n$. We note that $2i (2i - 1)!! = (2i + 1) (2i - 1)!! - (2i - 1)!! = (2i + 1)!! - (2i - 1)!!$. We thus have that
\begin{align*}
  1 + 2 \sum_{i = 1}^{n} i (2i - 1)!! & = 1 + \sum_{i = 1}^{n} \left( (2i + 1)!! - (2i - 1)!! \right) \\ 
	& = 1 + (2n + 1)!! - 1!! = (2n + 1)!! = 1 + 2k_n,
\end{align*}
proving our claim. This implies that for our specified sequence, we have that $a_{n + 1} = 2k_n + 1$.

We now show that for each $m$ with $0 \leq m \leq n + 1$, that Jacob can measure all of the positive integers from $m a_{n + 1} - k_n$ to $ma_{n + 1} + k_n$.  For each integer $i$ from $-k_n$ to $k_n$, the induction hypothesis tells us that Jacob can place the weights with masses $a_1, a_2, \dots, a_n$ on the scale in such a way that the difference between the total weight on the left hand side of the scale and the total weight on the right hand side of the scale is $i$. Jacob can then measure the weight $m a_{n + 1} + i$ by placing $m$ weights with mass $a_{n + 1}$ on the left hand side of the scale.

Since $(m + 1) a_{n + 1} - k_n - (m a_{n + 1} + k_n) = a_{n + 1} - 2k_n = 1$, this implies that Jacob can measure all of the positive integer weights from $0 a_{n + 1} - k_n$ to $(n + 1) a_{n + 1} + k_n$. We are thus finished if we have that $(n + 1) a_{n + 1} + k_n = k_{n + 1}$. But we have that
\[
  k_{n + 1} = \sum_{i = 1}^{n + 1} i (2i - 1)!! = k_n + (n + 1) (2n + 1)!! = (n + 1) a_{n + 1} + k_n,
\]
and we are done.


\vspace{6pt}
\item
\textit{Does there exist a natural number $n$ such that
\[
	1^{2018} + 2^{2018} + \cdots + n^{2018}
\]
is prime?}

We will prove that there is no such integer. Let
\[
	f(n) = 1^{2018} + 2^{2018} + \cdots + n^{2018}.
\]

Suppose that $p$ is a prime such that $p \mid n$. For any integer $k$ such that $0 < k < (p - 1)$, it is known that
\[
	1^k + 2^k + \cdots + p^k \equiv 0 \pmod p.
\]

To show this, let $g$ be a primitive root modulo $p$. Then we have that
\[
	1^k + 2^k + \cdots + {(p - 1)}^k \equiv g^k + g^{2k} + \cdots + g^{(p - 1)k} \equiv g^k \cdot \frac{g^{(p - 1)k} - 1}{g^k - 1}.
\]
Since $p$ divides $g^{(p - 1)k} - 1$, but $p$ does not divide $g^k - 1$, we see that this is congruent to $0$ modulo $p$.

Now using the division algorithm, we can write $2018 = q(p - 1) + r$ where $0 \leq r < (p - 1)$.

If $r \neq 0$, then we have that
\begin{align*}
	1^{2018} & + 2^{2018} + \cdots + n^{2018} \\
	& \equiv \underbrace{\left(1^{2018} + 2^{2018} + \cdots + p^{2018} \right) + \cdots + \left( 1^{2018} + 2^{2018} + \cdots + p^{2018} \right)}_{n/p \text{ times}} \\
	& \equiv \frac{n}{p} \left( 1^r + 2^r + \cdots + p^r \right) \\
	& \equiv 0 \pmod p
\end{align*}
and so $f(n)$ is divisible by $p$. Since we clearly have that $f(n) > p$, we see that $f(n)$ is not prime.

Thus if $f(n)$ is prime, and $p \mid n$, then we must have that $p - 1 \mid 2018$, and so $p - 1 \in \{1, 2, 1009, 2018\}$. Since $p$ is prime, this implies that $p \in \{2, 3\}$.

Now suppose that $p^2$ divides $n$. Then a similar argument to earlier shows that
\[
	f(n) \equiv \frac{n}{p} \left( 1^{2018} + 2^{2018} + \cdots + p^{2018} \right) \pmod p
\]
and since $n/p$ is divisible by $p$, we have that $f(n)$ is divisible by $p$ and hence is not prime.

Thus we must have that $n$ is square-free, and so $n \in \{1, 2, 3, 6\}$. Checking these values individually, we find that $f(1) = 1$ is not prime, $f(2) = 2^{2018} + 1$ is divisible by $5$, $f(3)$ is even, and $f(6)$ is divisible by $7$.


\vspace{6pt}
\item % Modified version of IMO1999 Problem 2
\textit{Fix a natural number $n \geq 2$. Find the smallest constant $C$ such that
\[
	 \sum_{1 \leq i < j \leq n} x_i x_j (3x_i^2 + x_j^2)(x_i^2 + 3x_j^2) \leq C {\left( \sum_{i = 1}^{n} x_i \right)}^6
\]
for all non-negative real numbers $x_1, x_2, \dots, x_n$. For this value of $C$, when does equality occur?}

Equivalently, we wish to determine the maximum value of the function
\[
	f(x_1, x_2, \dots, x_n) = \sum_{1 \leq i < j \leq n} x_i x_j (3x_i^2 + x_j^2)(x_i^2 + 3x_j^2)
\]
subject to the constraint $x_1 + x_2 + \cdots + x_n = 1$.

We claim that as long as there are at least three non-zero values among the $x_i$ then it is possible to increase the value of $f$ while maintaining the given constraint.

Suppose WLOG that $0 < x_1 \leq x_2 \leq x_3$. Consider the values $x_i^\prime$ where $x_1^\prime = x_1 + x_2$, $x_2^\prime = 0$, and $x_i^\prime = x_i$ for $i \geq 3$. We claim that
\[
	f(x_1, x_2, \dots, x_n) < f(x_1^\prime, x_2^\prime, \dots, x_n^\prime)
\]

The only terms in the sum defining $f(x_1, x_2, \dots, x_n)$ which differ from those defining $f(x_1^\prime, x_2^\prime, \dots x_n^\prime)$ are those for which either $i$ or $j$ is equal to $1$ or $2$, and so it is enough to prove that
\begin{align*}
	& \sum_{k = 2}^{n} x_1 x_k (3x_1^2 + x_k^2)(x_1^2 + 3x_k^2) + \sum_{k = 3}^{n} x_2 x_k (3x_2^2 + x_k^2)(x_2^2 + 3x_k^2) \\
	& <	\sum_{k = 2}^{n} x_1^\prime x_k^\prime (3{x_1^\prime}^2 + {x_k^\prime}^2)({x_1^\prime}^2 + 3{x_k^\prime}^2) + \sum_{k = 3}^{n} x_2^\prime x_k^\prime (3{x_2^\prime}^2 + {x_k^\prime}^2)({x_2^\prime}^2 + 3{x_k^\prime}^2)
\end{align*}
which is equivalent to
\begin{align*}
	& \sum_{k = 2}^{n} x_1 x_k (3x_1^2 + x_k^2)(x_1^2 + 3x_k^2) + \sum_{k = 3}^{n} x_2 x_k (3x_2^2 + x_k^2)(x_2^2 + 3x_k^2) \\
	& < \sum_{k = 3}^{n} (x_1 + x_2) x_k (3{(x_1 + x_2)}^2 + x_k^2)({(x_1 + x_2)}^2 + x_k^2).
\end{align*}

For $k > 3$, we have that
\[
	x_1 x_k (3{(x_1 + x_2)}^2 + x_k^2)({(x_1 + x_2)}^2 + x_k^2) > x_1 x_k (3x_1^2 + x_k^2)(x_1^2 + 3x_k^2)
\]
and similarly
\[
	x_2 x_k (3{(x_1 + x_2)}^2 + x_k^2)({(x_1 + x_2)}^2 + x_k^2) > x_2 x_k (3x_2^2 + x_k^2)(x_2^2 + 3x_k^2)
\]
and thus
\begin{align*}
	& \sum_{k = 4}^{n} (x_1 + x_2) x_k (3{(x_1 + x_2)}^2 + x_k^2)({(x_1 + x_2)}^2 + x_k^2) \\
	& > \sum_{k = 4}^{n} x_1 x_k (3x_1^2 + x_k^2)(x_1^2 + 3x_k^2) + \sum_{k = 4}^{n} x_2 x_k (3x_2^2 + x_k^2)(x_2^2 + 3x_k^2).
\end{align*}

It is thus enough to prove that
\begin{align*}
	& (x_1 + x_2) x_3 (3{(x_1 + x_2)}^2 + x_3^2)({(x_1 + x_2)}^2 + 3x_3^2) \\
	& > x_1 x_2 (3x_1^2 + x_2^2)(x_1^2 + 3x_2^2) + x_1 x_3 (3x_1^2 + x_3^2)(x_1^2 + 3x_3^2) \\
	& + x_2 x_3 (3x_2^2 + x_3^2)(x_2^2 + 3x_3^2)
\end{align*}
which simplifies to
\begin{align*}
	& x_1 x_3 (3{(x_1 + x_2)}^4 + 10 x_3^2 {(x_1 + x_2)}^2 + 3x_3^4 - 3x_1^4 - 10x_1^2 x_3^2 - 3x_3^4) \\
	& + x_2 x_3 (3{(x_1 + x_2)}^4 + 10 x_3^2 {(x_1 + x_2)}^2 + 3x_3^4 - 3x_2^4 - 10x_2^2 x_3^2 - 3x_3^4) \\
	& > x_1 x_2 (3x_1^4 + 10x_1^2 x_2^2 + 3x_2^4)
\end{align*}

Note that
\[
	3{(x_1 + x_2)}^4 - 3x_1^4 > 3x_2^4
\]
and
\[
	10 x_3^2 {(x_1 + x_2)}^2 - 10x_1^2 x_3^2 > 10 x_2^2 x_3^2 \geq 10 x_1^2 x_2^2
\]
where in the last inequality we use that $x_2 \leq x_3$.

We thus have that
\begin{align*}
	& x_1 x_3 (3{(x_1 + x_2)}^4 + 10 x_3^2 {(x_1 + x_2)}^2 + 3x_3^4 - 3x_1^4 - 10x_1^2 x_3^2 - 3x_3^4) \\
	& > x_1 x_3 (3x_2^4 + 10x_1^2 x_2^2) \geq x_1 x_2 (3x_2^4 + 10x_1^2 x_2^2).
\end{align*}

Similarly
\begin{align*}
	& x_2 x_3 (3{(x_1 + x_2)}^4 + 10 x_3^2 {(x_1 + x_2)}^2 + 3x_3^4 - 3x_2^4 - 10x_2^2 x_3^2 - 3x_3^4) \\
	& > x_1 x_2 (3x_1^4 + 10x_1^2 x_2^2)
\end{align*}
and the result follows.

Now for any set of values $x_1, x_2, \dots, x_n$ such that $x_1 + \cdots + x_n = 1$, we can apply the above procedure to increase the value of $f(x_1, x_2, \dots, x_n)$. The procedure increases the number of the $x_i$ which are equal to $0$ on each step, and so after a finite number of applications of the procedure, we obtain new values for the $x_i$ such that at most two of the $x_i$ are non-zero, and such that $f$ evaluated at the new values of $x_i$ is strictly larger than the original value of $f$.

It is thus enough to consider the case where at most two of the $x_i$ are non-zero. We thus wish to find the smallest constant $C$ such that
\[
	x y (3x^2 + y)(x^2 + 3y^2) < C{(x + y)}^6
\]
for all non-negative reals $x$ and $y$. We claim that the best such value for $C$ is $1/4$. Notice that if $x = y$, then we have equality, so we need only show that
\[
	{(x + y)}^6 \geq 4xy(3x^2 + y^2)(x^2 + 3y^2)
\]
and determine when equality holds.

But
\[
	{(x + y)}^6 \geq 4xy(3x^2 + y^2)(x^2 + 3y^2)
\]
is equivalent to
\[
	{(x - y)}^6 \geq 0
\]
and so we see that the inequality does hold for $C = 1/4$, and that equality occurs if and only if $x = y$. Since equality does occur, $C = 1/4$ is the best possible constant. In terms of the $x_i$, equality holds if and only if two of the $x_i$ are equal, and the rest are $0$.


\vspace{6pt}
\item 
\textit{Let $ABC$ be a triangle with circumcircle $\Omega$. Let $P$, $Q$ be 2 points not on $\Omega$ such that the line $PQ$ passes through the centre of $\Omega$. Let $D$, $E$, $F$ be the feet of the perpendiculars from $P$ to $BC$, $CA$, $AB$, and let $X$, $Y$, $Z$ be the feet of the perpendiculars from $Q$ onto the same sides, respectively. Prove that the perpendiculars from $D$, $E$, $F$ to $YZ$, $ZX$, $XY$ are concurrent.}

Let line $PQ$ pass through the circumcenter $O$ of $\Omega$. Fix point $Q$ and move $P$ along this line. The perpendiculars from $D$, $E$, $F$ to sidelines of $XYZ$ move uniformly and remain self-parallel, so their common points move along some lines. When $P$ coincides
with $O$ or $Q$, the three perpendiculars concur. Hence this is correct for all $P$.\\\\
The above argument yields that for point $Q$ fixed, the locus of points $P$ such that the perpendiculars concur is line $OQ$ or the whole plane. Supposing the second case, take point $C$ for $P$. Then $D$, $E$ coincide with $C$, and $F$ is the foot of the altitude from $C$. Since the three perpendiculars concur, we have $XY||AB$, and so $Q$ lies on $OC$. Taking now another vertex for $P$, we obtain that $Q$ coincides with $O$.

\end{enumerate}


\vfill
\textbf{\Large Email submission guidelines}
\begin{itemize}
	\item Email your solutions to \verb!samf.training.assignments@gmail.com!.
	\item Submit each question in a single separate PDF file (with multiple pages if necessary), with your name and the question number written on each page.
	\item If you take photographs of your work, use a document scanner such as CamScanner to convert to PDF.
	\item If you have multiple PDF files for a question, combine them using software such as PDFsam.
\end{itemize}

\end{document}
